\PassOptionsToPackage{unicode}{hyperref}
\PassOptionsToPackage{hyphens}{url}

\documentclass[]{article}


\usepackage{color}
\usepackage{graphicx}

\usepackage{fancyvrb}
\newcommand{\VerbBar}{|}
\newcommand{\VERB}{\Verb[commandchars=\\\{\}]}
\DefineVerbatimEnvironment{Highlighting}{Verbatim}{commandchars=\\\{\}}
% Add ',fontsize=\small' for more characters per line
\usepackage{framed}
\definecolor{shadecolor}{RGB}{248,248,248}
\newenvironment{Shaded}{\begin{snugshade}}{\end{snugshade}}
\newcommand{\AlertTok}[1]{\textcolor[rgb]{0.94,0.16,0.16}{#1}}
\newcommand{\AnnotationTok}[1]{\textcolor[rgb]{0.56,0.35,0.01}{\textbf{\textit{#1}}}}
\newcommand{\AttributeTok}[1]{\textcolor[rgb]{0.13,0.29,0.53}{#1}}
\newcommand{\BaseNTok}[1]{\textcolor[rgb]{0.00,0.00,0.81}{#1}}
\newcommand{\BuiltInTok}[1]{#1}
\newcommand{\CharTok}[1]{\textcolor[rgb]{0.31,0.60,0.02}{#1}}
\newcommand{\CommentTok}[1]{\textcolor[rgb]{0.56,0.35,0.01}{\textit{#1}}}
\newcommand{\CommentVarTok}[1]{\textcolor[rgb]{0.56,0.35,0.01}{\textbf{\textit{#1}}}}
\newcommand{\ConstantTok}[1]{\textcolor[rgb]{0.56,0.35,0.01}{#1}}
\newcommand{\ControlFlowTok}[1]{\textcolor[rgb]{0.13,0.29,0.53}{\textbf{#1}}}
\newcommand{\DataTypeTok}[1]{\textcolor[rgb]{0.13,0.29,0.53}{#1}}
\newcommand{\DecValTok}[1]{\textcolor[rgb]{0.00,0.00,0.81}{#1}}
\newcommand{\DocumentationTok}[1]{\textcolor[rgb]{0.56,0.35,0.01}{\textbf{\textit{#1}}}}
\newcommand{\ErrorTok}[1]{\textcolor[rgb]{0.64,0.00,0.00}{\textbf{#1}}}
\newcommand{\ExtensionTok}[1]{#1}
\newcommand{\FloatTok}[1]{\textcolor[rgb]{0.00,0.00,0.81}{#1}}
\newcommand{\FunctionTok}[1]{\textcolor[rgb]{0.13,0.29,0.53}{\textbf{#1}}}
\newcommand{\ImportTok}[1]{#1}
\newcommand{\InformationTok}[1]{\textcolor[rgb]{0.56,0.35,0.01}{\textbf{\textit{#1}}}}
\newcommand{\KeywordTok}[1]{\textcolor[rgb]{0.13,0.29,0.53}{\textbf{#1}}}
\newcommand{\NormalTok}[1]{#1}
\newcommand{\OperatorTok}[1]{\textcolor[rgb]{0.81,0.36,0.00}{\textbf{#1}}}
\newcommand{\OtherTok}[1]{\textcolor[rgb]{0.56,0.35,0.01}{#1}}
\newcommand{\PreprocessorTok}[1]{\textcolor[rgb]{0.56,0.35,0.01}{\textit{#1}}}
\newcommand{\RegionMarkerTok}[1]{#1}
\newcommand{\SpecialCharTok}[1]{\textcolor[rgb]{0.81,0.36,0.00}{\textbf{#1}}}
\newcommand{\SpecialStringTok}[1]{\textcolor[rgb]{0.31,0.60,0.02}{#1}}
\newcommand{\StringTok}[1]{\textcolor[rgb]{0.31,0.60,0.02}{#1}}
\newcommand{\VariableTok}[1]{\textcolor[rgb]{0.00,0.00,0.00}{#1}}
\newcommand{\VerbatimStringTok}[1]{\textcolor[rgb]{0.31,0.60,0.02}{#1}}
\newcommand{\WarningTok}[1]{\textcolor[rgb]{0.56,0.35,0.01}{\textbf{\textit{#1}}}}
% Add the geometry package and set smaller margins
\usepackage[margin=1in]{geometry}

\usepackage{amsmath,amssymb}
\usepackage{lmodern}
\usepackage{iftex}
\ifPDFTeX
  \usepackage[T1]{fontenc}
  \usepackage[utf8]{inputenc}
  \usepackage{textcomp} % provide euro and other symbols
\else % if luatex or xetex
  \usepackage{unicode-math} % this also loads fontspec
  \defaultfontfeatures{Scale=MatchLowercase}
  \defaultfontfeatures[\rmfamily]{Ligatures=TeX,Scale=1}
\fi
\usepackage{braket}
\usepackage{fancyhdr}
\usepackage{listings}
\usepackage{framed}
\lstnewenvironment{Highlighting}{}{}
\usepackage{xcolor}
% Use upquote if available, for straight quotes in verbatim environments
\IfFileExists{upquote.sty}{\usepackage{upquote}}{}

\IfFileExists{microtype.sty}{
  \usepackage[]{microtype}
  \UseMicrotypeSet[protrusion]{basicmath} % disable protrusion for tt fonts
}{}

\makeatletter
\@ifundefined{KOMAClassName}{% if non-KOMA class
  \IfFileExists{parskip.sty}{
    \usepackage{parskip}
  }{% else
    \setlength{\parindent}{0pt}
    \setlength{\parskip}{6pt plus 2pt minus 1pt}
  }
}{% if KOMA class
  \KOMAoptions{parskip=half}
}
\makeatother

\usepackage{xcolor}
\usepackage{longtable,booktabs,array}
\usepackage{calc} % for calculating minipage widths

% Correct order of tables after \paragraph or \subparagraph
\usepackage{etoolbox}
\makeatletter
\patchcmd\longtable{\par}{\if@noskipsec\mbox{}\fi\par}{}{}
\makeatother

% Allow footnotes in longtable head/foot
\IfFileExists{footnotehyper.sty}{\usepackage{footnotehyper}}{\usepackage{footnote}}
\makesavenoteenv{longtable}

\setlength{\emergencystretch}{3em} % prevent overfull lines
\providecommand{\tightlist}{%
  \setlength{\itemsep}{0pt}\setlength{\parskip}{0pt}}
\setcounter{secnumdepth}{-\maxdimen} % remove section numbering

\usepackage{framed}
\usepackage{listings}

\ifLuaTeX
  \usepackage{selnolig} % disable illegal ligatures
\fi

\IfFileExists{bookmark.sty}{\usepackage{bookmark}}{\usepackage{hyperref}}
\IfFileExists{xurl.sty}{\usepackage{xurl}}{} % add URL line breaks if available
\urlstyle{same}
\hypersetup{
  hidelinks,
  pdfcreator={LaTeX via pandoc}
}

% Custom header and footer
\pagestyle{fancy}
\fancyhead[l]{Azal Amer}
\fancyhead[c]{Quantum 1 HW  \#2}
\fancyhead[r]{\today}
\fancyfoot[c]{\thepage}
\renewcommand{\headrulewidth}{.2pt}
\setlength{\headheight}{15pt}

\begin{document}
\definecolor{shadecolor}{rgb}{0.9, 0.9, 0.9}

\hypertarget{problem-1}{%
\section{Problem 1}\label{problem-1}}

In the \(\left\{ \ket{1},\ket{2} \right\}\) basis, we define the
representations in \(\mathbb{R}^{2}\) as\\
\[
\begin{align}
\ket{1}  = \begin{pmatrix}
1 \\
0
\end{pmatrix}  &  & \ket{2} =\begin{pmatrix}
0 \\
1
\end{pmatrix}
\end{align}
\] Each component \[
\begin{align}
H = H_{11}\begin{pmatrix}
1 \\
0 
\end{pmatrix}\begin{pmatrix}
1 & 0
\end{pmatrix}+H_{22}\begin{pmatrix}
0 \\
1
\end{pmatrix}\begin{pmatrix}
0 & 1
\end{pmatrix}+H_{12}\left[ \begin{pmatrix}
1 \\
0
\end{pmatrix}\begin{pmatrix}
0 & 1
\end{pmatrix}+\begin{pmatrix}
0 \\
1
\end{pmatrix} \begin{pmatrix}
1 & 0
\end{pmatrix}\right]   
\\
=H_{11}\begin{pmatrix}
1 & 0 \\
0 & 0
\end{pmatrix}+H_{22}\begin{pmatrix}
0 & 0 \\
0 & 1
\end{pmatrix}+H_{12}\left[ \begin{pmatrix}
0 & 1 \\
0 & 0
\end{pmatrix} +\begin{pmatrix}
0 & 0 \\
1 & 0
\end{pmatrix}\right]  \\
H=\begin{pmatrix}
H_{11} & H_{12} \\
H_{12} & H_{22}
\end{pmatrix}
\end{align}
\]

\[
\begin{align}
 \lambda_{1}=\frac{H_{11}}{2} + \frac{H_{22}}{2} - \frac{\sqrt{H_{11}^{2} - 2 H_{11} H_{22} + 4 H_{12}^{2} + H_{22}^{2}}}{2}  &  & \vec{v_{1}}=\begin{pmatrix}
-H_{22}+\lambda_{1} \\
H_{12}
\end{pmatrix}\\
\lambda_{2} = \frac{H_{11}}{2} + \frac{H_{22}}{2} + \frac{\sqrt{H_{11}^{2} - 2 H_{11} H_{22} + 4 H_{12}^{2} + H_{22}^{2}}}{2} &  & \vec{v_{2}}=\begin{pmatrix}
-H_{22}+\lambda_{2} \\
H_{12}
\end{pmatrix}
\end{align}
\]

Our normalized eigenkets are \[
\begin{align}
\ket{v_{1}}  = \begin{pmatrix}\frac{- H_{22} + \lambda_{1}}{\sqrt{\left(H_{22} - \lambda_{1}\right)^{2} + H_{12}^{2}}}\\\frac{H_{12}^{2}}{\sqrt{\left(H_{22} - \lambda_{1}\right)^{2} + 1}}\end{pmatrix} \\
\ket{v_{2}}  = \left[\begin{matrix}\frac{- H_{22} + \lambda_{2}}{\sqrt{\left(H_{22} - \lambda_{2}\right)^{2} + H_{12}^{2}}}\\ \frac{H_{12}^{2}}{\sqrt{\left(H_{22} - \lambda_{2}\right)^{2} + H_{12}^{2}}}\end{matrix}\right]
\end{align} 
\]

With these eigenvalues, we know that if \(H_{12}=0\), the eigenvalues
should be \(H_{11},H_{22}\) Plugging in for \(\lambda_{1}\) \[
\begin{align}
\frac{H_{11}}{2} + \frac{H_{22}}{2} + \frac{\sqrt{H_{11}^{2} - 2 H_{11} H_{22} + H_{22}^{2}}}{2} \\
\frac{H_{11}}{2} + \frac{H_{22}}{2}+{\frac{\sqrt{ \left( H_{11}-H_{22} \right) ^{2} }}{2}} \\
\frac{1}{2}\left( H_{11}+H_{22}+\left| H_{11}-H_{22} \right|  \right)  \\
\lambda_{1} = \begin{cases} \left( H_{11}-H_{22}>0 \right) :
H_{11} \\
\left( H_{11}-H_{22}<0 \right):H_{22}
\end{cases}
\end{align}
\] Repeating this same step with \(\lambda_{2}\) yields the same two
possible answers due to the absolute value term. This is what we expect.

\hypertarget{problem-2}{%
\section{Problem 2}\label{problem-2}}

\hypertarget{a}{%
\subsection{a)}\label{a}}

\[
H = \Delta \left(\ket{R}\bra{L}+\ket{L}\bra{R} \right) = \Delta \begin{pmatrix}
0 & 1 \\
1 & 0
\end{pmatrix} 
\] The eigenvectors of this matrix and energy eigenvalues, normalized
are \[
\begin{align}
\lambda_{1} = \Delta  &  & \ket{v_{1}}  = \frac{1}{\sqrt{ 2 }}\begin{pmatrix}
1 \\
1
\end{pmatrix} \\
\lambda_{2} = -\Delta  &  & \ket{v_{2}}  = \frac{1}{\sqrt{ 2 }}\begin{pmatrix}
-1 \\
1
\end{pmatrix}
\end{align}
\]

\hypertarget{b}{%
\subsection{b)}\label{b}}

\[
\ket{\alpha}  = a\ket{v_{1}} + b\ket{v_{2}}  
\] This is our initial state. We know that it evolves as a function of
time, so \(a,b\) are both time-dependent. Rewriting \(\ket{\alpha}\) to
matrix form in the \(\left\{ \ket{R},\ket{L} \right\}\) basis gives \[
\ket{\alpha} =\frac{1}{\sqrt{ 2 }}\begin{pmatrix}
a(t)-b(t) \\
a(t)+b(t)
\end{pmatrix}
\] From here, we plug \(\ket{\alpha}\) the Schrödinger equation. \[
\begin{align}
i\hbar \frac{d}{dt} \ket{\alpha(t)} =H\ket{\alpha(t)}  \\
\dot{\ket{a(t)}} = \frac{1}{\sqrt{ 2 }}\begin{pmatrix}
\dot{a(t)}-\dot{b(t)} \\
\dot{a(t)}+\dot{b(t)} \\
\end{pmatrix} \\
H\ket{\alpha} = \frac{1}{\sqrt{ 2 }}\begin{pmatrix}
0 & 1 \\
1 & 0
\end{pmatrix}\begin{pmatrix}
a(t)-b(t) \\
a(t)+b(t)
\end{pmatrix}=\frac{1}{\sqrt{ 2 }}\begin{pmatrix}
a(t)+b(t) \\
a(t)-b(t)
\end{pmatrix} \\
\begin{pmatrix}
\dot{a(t)}-\dot{b(t)} \\
\dot{a(t)}+\dot{b(t)} \\
\end{pmatrix} =\frac{1}{i\hbar}\begin{pmatrix}
a(t)+b(t) \\
a(t)-b(t)
\end{pmatrix}
\end{align}
\]

\begin{Shaded}
\begin{Highlighting}[]
\NormalTok{import sympy as sp}

\NormalTok{t = sp.symbols(\textquotesingle{}t\textquotesingle{})}
\NormalTok{hbar = sp.symbols(\textquotesingle{}hbar\textquotesingle{}, real=True)}
\NormalTok{a = sp.Function(\textquotesingle{}a\textquotesingle{})(t)}
\NormalTok{b = sp.Function(\textquotesingle{}b\textquotesingle{})(t)}
\NormalTok{delta = sp.symbols(\textquotesingle{}Delta\textquotesingle{},real = True)}

\NormalTok{lhs = sp.I * hbar * sp.Matrix([[sp.diff(a, t) {-} sp.diff(b, t)], [sp.diff(a, t) + sp.diff(b, t)]])}
\NormalTok{rhs = (delta) * sp.Matrix([[a + b], [a {-} b]])}


\NormalTok{sol = sp.dsolve(lhs {-} rhs)}

\NormalTok{print(sp.latex(sol))}
\end{Highlighting}
\end{Shaded}

This then gives us a solution of

\[
\ket{\alpha(t)} =c_{1} e^{\frac{i \Delta t}{\hbar}}\ket{v_{2}} +c_{2}e^{ -i\Delta t/\hbar }\ket{v_{1}} 
\]

Because we don't know the state's initial conditions, this is the
simplified form within the basis formed by the matrix.

\hypertarget{c}{%
\subsection{c)}\label{c}}

To solve this part of the question, we need to find the values of
\(c_{1,2}\). We know that at \(t=0\), \(\ket{\alpha}=\ket{R}\). We will
use \(\ket{\alpha(t=0)}\) to solve for \(c_{1,2}\) Before that though,
we need to characterize \(\ket{R}\) as a linear combination of
\(\left\{ \ket{v_{1}},\ket{v_{2}} \right\}\) . In this case \[
\begin{align} 
\begin{pmatrix}
1 \\
0 
\end{pmatrix} = \frac{1}{\sqrt{ 2 }}\left( \frac{1}{\sqrt{ 2 }}\begin{pmatrix}
1 \\
1
\end{pmatrix} +\frac{1}{\sqrt{ 2 }}\begin{pmatrix}
1 \\
-1
\end{pmatrix}\right) \\
\ket{R}  = \frac{1}{\sqrt{ 2 }} \left( \ket{v_{1}} +\ket{v_{2}}  \right) 
\end{align}
\]

\[
\ket{\alpha(0)} =c_{1}\ket{R} +c_{2}\ket{L}  = \ket{R} 
\] Therefore in our initial system,

\[
\begin{align}
c_{1} = \frac{1}{\sqrt{ 2 }} \\
c_{2}=\frac{1}{\sqrt{ 2 }}
\end{align}
\]

Our probability of observing \(\ket{L}\) comes from Born's rule, which
is the amplitude of the ket squared. Rewriting \(\ket{\alpha(t)}\) in
terms of \(\ket{R},\ket{L}\) is

\[\frac{1}{\sqrt{ 2 }}
\begin{pmatrix}
1 & 1 \\
1 & -1
\end{pmatrix} \frac{1}{\sqrt{2}}\begin{pmatrix}
e^{ i\Delta  t/\hbar } \\
e^{ -i\Delta t/\hbar }
\end{pmatrix}=\begin{pmatrix}\cos{\left(\frac{ \Delta t}{ \hbar} \right)}\\i \sin{\left(\frac{ \Delta t}{ \hbar} \right)}\end{pmatrix}
\]

The probability over time is finally just \[
\mathbb{P}_{\ket{L} }(t) = \sin^{2} \left( \frac{\Delta t}{\hbar} \right) 
\]

\hypertarget{d}{%
\subsection{d)}\label{d}}

I think I already worked this out in part a, part b, and part c. \[
\ket{\psi}  = \cos \left( \frac{\Delta t}{\hbar} \right)\ket{R}  +i\sin \left( \frac{\Delta t}{\hbar} \right) \ket{L} 
\]

\hypertarget{problem-3}{%
\section{Problem 3}\label{problem-3}}

\hypertarget{a-1}{%
\subsection{a)}\label{a-1}}

\emph{Credit to Ayden Gertiser for help with this problem}

\includegraphics{../../../Supplemental Files/images/Pasted image 20240922202151.png|100}

We represent the vector \(\hat{n}\) in the
\(\left\{ z_{+},z_{-} \right\}\) basis \[
\hat{n} = \left( \sin \gamma,0,\cos \gamma \right) 
\] To find it's probability in the \(X\) basis, we use the fact found in
HW1, that the \(\frac{\hbar}{2}\) eigenvector is
\(\left( \cos \frac{\gamma}{2},\sin \frac{\gamma}{2} \right)\).
Representing this in the \(X\) basis is found by just \[
\begin{align}
\cos \frac{\gamma}{2}\ket{z_{+}} +\sin \frac{\gamma}{2}\ket{z_{+}} = \frac{1}{\sqrt{ 2 }} \cos \frac{\gamma}{2}\left(\ket{x_{+}  }+\ket{x_{-}} )\right)  + \frac{1}{\sqrt{ 2 }} \sin \frac{\gamma}{2} \left(  \ket{x_{+}  }-\ket{x_{-}} )\right) 
\end{align}
\] From this, \[
\mathbb{P}\left( \frac{\hbar}{2} \right) = \frac{1}{2}\left(  \cos \frac{\gamma}{2}+ \sin \frac{\gamma}{2} \right) ^{2}
\]

\hypertarget{b-1}{%
\subsection{b)}\label{b-1}}

\[
S\cdot\vec{n} = S_{x}\sin\gamma+S_{z}\cos\gamma
\] Whichever vector has the eigenvalue of \(\frac{\hbar}{2}\) in these
operators, is the state of our system. Before that though, we can reduce
the expression to the below

\[
\begin{align}
\left<(S_{x}-\left<S_{x}\right>)^{2} \right>  \\
\left<S_{x}^{2}-2S_{x}\left<S_{x}\right>+\left<S_{x}\right>^{2}\right>  \\
\left<S_{x}^{2}\right> -2\left<S_{x}\right>^{2}+\left<S_{x}\right>^{2}\\ 
=\left<S_{x}^{2}\right> -\left<S_{x}\right>^{2}\\
\end{align}
\]

\begin{Shaded}
\begin{Highlighting}[]
\NormalTok{Sx = sp.Matrix([}
\NormalTok{[0,1],}
\NormalTok{[1,0]}
\NormalTok{])}
\NormalTok{Sz = sp.Matrix([}
\NormalTok{[1,0],}
\NormalTok{[0,{-}1]}
\NormalTok{])}
\NormalTok{gamma = sp.symbols(\textquotesingle{}gamma\textquotesingle{},real = True)}
\NormalTok{observable = Sx*sp.sin(gamma)+Sz*sp.cos(gamma)}
\NormalTok{eigenvect = observable.eigenvects()[1][2][0]}
\NormalTok{print(sp.latex(eigenvect))}
\end{Highlighting}
\end{Shaded}

Our eigenvector is then \[
\ket{\alpha}  = k(-\sin(\gamma)\ket{0} +(\cos(\gamma)-1)\ket{1} )
\] Where \(k\) is \[
k=\sqrt{\left|{\frac{\sin{\left(\gamma \right)}}{\cos{\left(\gamma \right)} - 1}}\right|^{2} + 1}=\sqrt{ 2(1-\cos \gamma) }
\]

\[
\ket{\alpha} = \frac{1}{\sqrt{ 2(1-\cos \gamma) }}(-\sin(\gamma)\ket{0} +(\cos(\gamma)-1)\ket{1} )
\] To find our expected values, we'll start with \(S_{x}\) \[
\bra{\alpha} S_{x}\ket{\alpha} 
\]

\begin{Shaded}
\begin{Highlighting}[]
\NormalTok{gamma = sp.symbols(\textquotesingle{}gamma\textquotesingle{},real = True)}
\NormalTok{normFac = 1/(sp.sqrt(2*(1{-}sp.cos(gamma))))}
\NormalTok{vec = sp.Matrix([{-}sp.sin(gamma),sp.cos(gamma){-}1])}
\NormalTok{normVec = normFac*vec}
\NormalTok{expecSx = (normVec.T*Sx*normVec)[0]}
\NormalTok{print(sp.latex(expecSx))}
\end{Highlighting}
\end{Shaded}

\[
- \frac{2 \left(\cos{\left(\gamma \right)} - 1\right) \sin{\left(\gamma \right)}}{2 - 2 \cos{\left(\gamma \right)}}
\]

Then for \(S_{x}^{2}\)

\begin{Shaded}
\begin{Highlighting}[]
\NormalTok{gamma = sp.symbols(\textquotesingle{}gamma\textquotesingle{},real = True)}
\NormalTok{normFac = 1/(sp.sqrt(2*(1{-}sp.cos(gamma))))}
\NormalTok{vec = sp.Matrix([{-}sp.sin(gamma),sp.cos(gamma){-}1])}
\NormalTok{normVec = normFac*vec}
\NormalTok{expecSx2 =(normVec.T*(Sx**2)*normVec)[0]}
\NormalTok{print(sp.latex(expecSx2))}
\end{Highlighting}
\end{Shaded}

\[
\frac{\left(\cos{\left(\gamma \right)} - 1\right)^{2}}{2 - 2 \cos{\left(\gamma \right)}} + \frac{\sin^{2}{\left(\gamma \right)}}{2 - 2 \cos{\left(\gamma \right)}}
\] Then our difference becomes

\begin{Shaded}
\begin{Highlighting}[]
\NormalTok{dispersion = expecSx2 {-} expecSx**2}
\NormalTok{print(sp.latex(sp.simplify(dispersion.expand())))}
\end{Highlighting}
\end{Shaded}

Then our final dispersion is just \[
\left\langle \left(S_x - \langle S_x \rangle\right)^2 \right\rangle = \frac{\hbar^{2}}{4}\cos^{2}{\left(\gamma \right)}
\]

\hypertarget{c-1}{%
\subsection{c)}\label{c-1}}

Repeating this for \(\ket{z_{+}}\) gives us

\(S_{x}\)

\begin{Shaded}
\begin{Highlighting}[]
\NormalTok{gamma = sp.symbols(\textquotesingle{}gamma\textquotesingle{},real = True)}
\NormalTok{normFac = 1}
\NormalTok{vec = sp.Matrix([1,0])}
\NormalTok{normVec = normFac*vec}
\NormalTok{expecSx = (normVec.T*Sx*normVec)[0]}
\NormalTok{print(sp.latex(expecSx))}
\end{Highlighting}
\end{Shaded}

\[
- \frac{2 \left(\cos{\left(\gamma \right)} - 1\right) \sin{\left(\gamma \right)}}{2 - 2 \cos{\left(\gamma \right)}}
\]

Then for \(S_{x}^{2}\)

\begin{Shaded}
\begin{Highlighting}[]
\NormalTok{gamma = sp.symbols(\textquotesingle{}gamma\textquotesingle{},real = True)}
\NormalTok{normFac = 1}

\NormalTok{normVec = normFac*vec}
\NormalTok{expecSx2 =(normVec.T*(Sx**2)*normVec)[0]}
\NormalTok{print(sp.latex(expecSx2))}
\end{Highlighting}
\end{Shaded}

\[
\frac{1}{2 - 2 \cos{\left(\gamma \right)}}
\] Then our difference becomes

\begin{Shaded}
\begin{Highlighting}[]
\NormalTok{dispersion = expecSx2 {-} expecSx**2}
\NormalTok{print(sp.latex(sp.simplify(dispersion.expand())))}
\end{Highlighting}
\end{Shaded}

\[
\left<\left( \Delta S_{x} \right) ^{2}\right>  = \frac{\hbar^{2}}{4}
\] For the \(S_{y}\) observable

\begin{Shaded}
\begin{Highlighting}[]
\NormalTok{Sy = sp.Matrix([}
\NormalTok{[0,{-}sp.I],}
\NormalTok{[sp.I,0]}
\NormalTok{])}
\NormalTok{gamma = sp.symbols(\textquotesingle{}gamma\textquotesingle{},real = True)}
\NormalTok{normFac = 1}
\NormalTok{vec = sp.Matrix([1,0])}
\NormalTok{normVec = normFac*vec}
\NormalTok{expecSy = (normVec.T*Sy*normVec)[0]}
\NormalTok{print(sp.latex(expecSy))}
\NormalTok{gamma = sp.symbols(\textquotesingle{}gamma\textquotesingle{},real = True)}
\NormalTok{normFac = 1}

\NormalTok{normVec = normFac*vec}
\NormalTok{expecSy2 =(normVec.T*(Sy**2)*normVec)[0]}
\NormalTok{print(sp.latex(expecSy2))}
\NormalTok{dispersion = expecSy2 {-} expecSy**2}
\NormalTok{print(sp.latex(sp.simplify(dispersion.expand())))}
\end{Highlighting}
\end{Shaded}

\[
\left<\left( \Delta S_{y} \right) ^{2}\right>  = \frac{\hbar^{2}}{4}
\]

\hypertarget{problem-4}{%
\section{Problem 4}\label{problem-4}}

\hypertarget{a-2}{%
\subsection{a)}\label{a-2}}

\begin{Shaded}
\begin{Highlighting}[]
\NormalTok{A, B = sp.symbols(\textquotesingle{}A B\textquotesingle{}, real=True)}
\NormalTok{H = sp.Matrix([ }
\NormalTok{[0, {-}sp.I*A, 0, 0], }
\NormalTok{[sp.I*A, 0, 0, 0],}
\NormalTok{[0, 0, 0, B], }
\NormalTok{[0, 0, B, 0] }
\NormalTok{])}
\NormalTok{print(sp.latex(H.eigenvects()))}
\end{Highlighting}
\end{Shaded}

\[
\begin{align}
\lambda_1 &= -A, & \mathbf{v}_1 &= \begin{bmatrix} i \\ 1 \\ 0 \\ 0 \end{bmatrix} \\[10pt]
\lambda_2 &= A, & \mathbf{v}_2 &= \begin{bmatrix} -i \\ 1 \\ 0 \\ 0 \end{bmatrix} \\[10pt]
\lambda_3 &= -B, & \mathbf{v}_3 &= \begin{bmatrix} 0 \\ 0 \\ -1 \\ 1 \end{bmatrix} \\[10pt]
\lambda_4 &= B, & \mathbf{v}_4 &= \begin{bmatrix} 0 \\ 0 \\ 1 \\ 1 \end{bmatrix}
\end{align}
\] Above is our spectrum of the matrix.

\hypertarget{b-2}{%
\subsection{b)}\label{b-2}}

From our initial conditions, we define our vector \(\ket{\psi}\) to be
\[
\ket{\psi}  = \alpha_{1}(t)v_{1}+\alpha_{2}(t)v_{2}+\alpha_{3}(t)v_{3}+\alpha_{4}(t)v_{4} = \begin{pmatrix}
a_{1}(t) \\
a_{2}(t) \\ 
a_{3}(t) \\ 
a_{4}(t) \\
\end{pmatrix}
\] then \(\dot{\ket{\psi}}\) is \[
 \dot{\ket{\psi}}=
\begin{pmatrix}
\dot{a_{1}(t)} \\
\dot{a_{2}(t) }\\ 
\dot{a_{3}(t) }\\ 
\dot{a_{4}(t)} \\
\end{pmatrix}
\] Plugging this into our schrodinger equation gives

\[
i\hbar \begin{pmatrix}
\dot{a_{1}(t)} \\
\dot{a_{2}(t) }\\ 
\dot{a_{3}(t) }\\ 
\dot{a_{4}(t)} \\
\end{pmatrix} =  \begin{pmatrix}
0 & -iA & 0 & 0 \\
iA & 0 & 0 & 0 \\
0 & 0 & 0 & B \\
0 & 0 & B & 0
\end{pmatrix} \begin{pmatrix}
a_{1}(t) \\
a_{2}(t) \\ 
a_{3}(t) \\ 
a_{4}(t) \\
\end{pmatrix}
\]

\begin{Shaded}
\begin{Highlighting}[]
\NormalTok{import sympy as sp}

\NormalTok{\# Define symbols and functions}
\NormalTok{t = sp.Symbol(\textquotesingle{}t\textquotesingle{}, real=True)}
\NormalTok{A, B = sp.symbols(\textquotesingle{}A B\textquotesingle{}, real=True)}
\NormalTok{hbar = sp.Symbol(\textquotesingle{}hbar\textquotesingle{}, real=True, positive=True)}
\NormalTok{a1, a2, a3, a4 = sp.Function(\textquotesingle{}a1\textquotesingle{})(t), sp.Function(\textquotesingle{}a2\textquotesingle{})(t), sp.Function(\textquotesingle{}a3\textquotesingle{})(t), sp.Function(\textquotesingle{}a4\textquotesingle{})(t)}

\NormalTok{\# Define the system of differential equations}
\NormalTok{eq1 = sp.Eq(sp.I * hbar * a1.diff(t), {-}sp.I * A * a2)}
\NormalTok{eq2 = sp.Eq(sp.I * hbar * a2.diff(t), sp.I * A * a1)}
\NormalTok{eq3 = sp.Eq(sp.I * hbar * a3.diff(t), B * a4)}
\NormalTok{eq4 = sp.Eq(sp.I * hbar * a4.diff(t), B * a3)}

\NormalTok{\# Solve the system}
\NormalTok{solution = sp.dsolve([eq1, eq2, eq3, eq4])}

\NormalTok{print("Solution:")}
\NormalTok{for sol in solution:}
\NormalTok{    print(sp.latex(sol))}
\end{Highlighting}
\end{Shaded}

\[
\ket{\psi}  = 
\begin{pmatrix}
-i C_1 e^{-\frac{iAt}{\hbar}} + i C_2 e^{\frac{iAt}{\hbar}} \\[10pt]
C_1 e^{-\frac{iAt}{\hbar}} + C_2 e^{\frac{iAt}{\hbar}} \\[10pt]
C_3 e^{-\frac{iBt}{\hbar}} - C_4 e^{\frac{iBt}{\hbar}} \\[10pt]
C_3 e^{-\frac{iBt}{\hbar}} + C_4 e^{\frac{iBt}{\hbar}}
\end{pmatrix}
\] Now we want to evaluate our initial conditions at \(t=0\) Plugging
\(t=0\) to the solution of the Schrödinger equation gives \[
\ket{\psi}\bigg|_{t=0} = \begin{pmatrix}
-iC_{1}+iC_{2} \\
C_{1}+C_{2} \\
C_{3}-C_{4} \\
C_{3}+C_{4}
\end{pmatrix} = \begin{pmatrix}
0 \\
1 \\
1 \\
0
\end{pmatrix}
\] Then \(C_{1},C_{2}=\frac{1}{2}\) And
\(C_{3},C_{4} =\frac{1}{2},-\frac{1}{2}\) Thus in the \(\ket{z}\) basis
the solution is

\[
\ket{\psi(t)}  = 
\begin{pmatrix}
-\frac{i}{2} e^{-\frac{iAt}{\hbar}} + \frac{i}{2} e^{\frac{iAt}{\hbar}} \\
\frac{1}{2} e^{-\frac{iAt}{\hbar}} + \frac{1}{2} e^{\frac{iAt}{\hbar}} \\
\frac{1}{2} e^{-\frac{iBt}{\hbar}} + \frac{1}{2} e^{\frac{iBt}{\hbar}} \\
\frac{1}{2} e^{-\frac{iBt}{\hbar}} - \frac{1}{2} e^{\frac{iBt}{\hbar}}
\end{pmatrix}
\] However, after applying the \(H\) transformation, we get \[
\ket{\psi(t)}  = \frac{1}{2}e^{ iAt /\hbar}\ket{-A} +\frac{1}{2}e^{ -iAt /\hbar}\ket{A} +\frac{1}{2}e^{ -iBt /\hbar}\ket{B}-\frac{1}{2}e^{ iBt /\hbar}\ket{B}  
\]

\hypertarget{c-2}{%
\subsection{c)}\label{c-2}}

The probability of occurrence is just \[
aa^{\star} =  \frac{1}{2}e^{ iBt/h } \frac{1}{2}e^{ -iBt/h } = \frac{1}{4}
\] Then \[
P(E=B) = \frac{1}{4}
\]
\end{document}