\PassOptionsToPackage{unicode}{hyperref}
\PassOptionsToPackage{hyphens}{url}

\documentclass[]{article}


\usepackage{color}
\usepackage{fancyvrb}
\newcommand{\VerbBar}{|}
\newcommand{\VERB}{\Verb[commandchars=\\\{\}]}
\DefineVerbatimEnvironment{Highlighting}{Verbatim}{commandchars=\\\{\}}
% Add ',fontsize=\small' for more characters per line
\usepackage{framed}
\definecolor{shadecolor}{RGB}{248,248,248}
\newenvironment{Shaded}{\begin{snugshade}}{\end{snugshade}}
\newcommand{\AlertTok}[1]{\textcolor[rgb]{0.94,0.16,0.16}{#1}}
\newcommand{\AnnotationTok}[1]{\textcolor[rgb]{0.56,0.35,0.01}{\textbf{\textit{#1}}}}
\newcommand{\AttributeTok}[1]{\textcolor[rgb]{0.13,0.29,0.53}{#1}}
\newcommand{\BaseNTok}[1]{\textcolor[rgb]{0.00,0.00,0.81}{#1}}
\newcommand{\BuiltInTok}[1]{#1}
\newcommand{\CharTok}[1]{\textcolor[rgb]{0.31,0.60,0.02}{#1}}
\newcommand{\CommentTok}[1]{\textcolor[rgb]{0.56,0.35,0.01}{\textit{#1}}}
\newcommand{\CommentVarTok}[1]{\textcolor[rgb]{0.56,0.35,0.01}{\textbf{\textit{#1}}}}
\newcommand{\ConstantTok}[1]{\textcolor[rgb]{0.56,0.35,0.01}{#1}}
\newcommand{\ControlFlowTok}[1]{\textcolor[rgb]{0.13,0.29,0.53}{\textbf{#1}}}
\newcommand{\DataTypeTok}[1]{\textcolor[rgb]{0.13,0.29,0.53}{#1}}
\newcommand{\DecValTok}[1]{\textcolor[rgb]{0.00,0.00,0.81}{#1}}
\newcommand{\DocumentationTok}[1]{\textcolor[rgb]{0.56,0.35,0.01}{\textbf{\textit{#1}}}}
\newcommand{\ErrorTok}[1]{\textcolor[rgb]{0.64,0.00,0.00}{\textbf{#1}}}
\newcommand{\ExtensionTok}[1]{#1}
\newcommand{\FloatTok}[1]{\textcolor[rgb]{0.00,0.00,0.81}{#1}}
\newcommand{\FunctionTok}[1]{\textcolor[rgb]{0.13,0.29,0.53}{\textbf{#1}}}
\newcommand{\ImportTok}[1]{#1}
\newcommand{\InformationTok}[1]{\textcolor[rgb]{0.56,0.35,0.01}{\textbf{\textit{#1}}}}
\newcommand{\KeywordTok}[1]{\textcolor[rgb]{0.13,0.29,0.53}{\textbf{#1}}}
\newcommand{\NormalTok}[1]{#1}
\newcommand{\OperatorTok}[1]{\textcolor[rgb]{0.81,0.36,0.00}{\textbf{#1}}}
\newcommand{\OtherTok}[1]{\textcolor[rgb]{0.56,0.35,0.01}{#1}}
\newcommand{\PreprocessorTok}[1]{\textcolor[rgb]{0.56,0.35,0.01}{\textit{#1}}}
\newcommand{\RegionMarkerTok}[1]{#1}
\newcommand{\SpecialCharTok}[1]{\textcolor[rgb]{0.81,0.36,0.00}{\textbf{#1}}}
\newcommand{\SpecialStringTok}[1]{\textcolor[rgb]{0.31,0.60,0.02}{#1}}
\newcommand{\StringTok}[1]{\textcolor[rgb]{0.31,0.60,0.02}{#1}}
\newcommand{\VariableTok}[1]{\textcolor[rgb]{0.00,0.00,0.00}{#1}}
\newcommand{\VerbatimStringTok}[1]{\textcolor[rgb]{0.31,0.60,0.02}{#1}}
\newcommand{\WarningTok}[1]{\textcolor[rgb]{0.56,0.35,0.01}{\textbf{\textit{#1}}}}
% Add the geometry package and set smaller margins
\usepackage[margin=1in]{geometry}

\usepackage{amsmath,amssymb}
\usepackage{lmodern}
\usepackage{iftex}
\ifPDFTeX
  \usepackage[T1]{fontenc}
  \usepackage[utf8]{inputenc}
  \usepackage{textcomp} % provide euro and other symbols
\else % if luatex or xetex
  \usepackage{unicode-math} % this also loads fontspec
  \defaultfontfeatures{Scale=MatchLowercase}
  \defaultfontfeatures[\rmfamily]{Ligatures=TeX,Scale=1}
\fi
\usepackage{braket}
\usepackage{fancyhdr}

% Use upquote if available, for straight quotes in verbatim environments
\IfFileExists{upquote.sty}{\usepackage{upquote}}{}

\IfFileExists{microtype.sty}{
  \usepackage[]{microtype}
  \UseMicrotypeSet[protrusion]{basicmath} % disable protrusion for tt fonts
}{}

\makeatletter
\@ifundefined{KOMAClassName}{% if non-KOMA class
  \IfFileExists{parskip.sty}{
    \usepackage{parskip}
  }{% else
    \setlength{\parindent}{0pt}
    \setlength{\parskip}{6pt plus 2pt minus 1pt}
  }
}{% if KOMA class
  \KOMAoptions{parskip=half}
}
\makeatother

\usepackage{xcolor}
\usepackage{longtable,booktabs,array}
\usepackage{calc} % for calculating minipage widths

% Correct order of tables after \paragraph or \subparagraph
\usepackage{etoolbox}
\makeatletter
\patchcmd\longtable{\par}{\if@noskipsec\mbox{}\fi\par}{}{}
\makeatother

% Allow footnotes in longtable head/foot
\IfFileExists{footnotehyper.sty}{\usepackage{footnotehyper}}{\usepackage{footnote}}
\makesavenoteenv{longtable}

\setlength{\emergencystretch}{3em} % prevent overfull lines
\providecommand{\tightlist}{%
  \setlength{\itemsep}{0pt}\setlength{\parskip}{0pt}}
\setcounter{secnumdepth}{-\maxdimen} % remove section numbering

\usepackage{framed}
\usepackage{listings}

\ifLuaTeX
  \usepackage{selnolig} % disable illegal ligatures
\fi

\IfFileExists{bookmark.sty}{\usepackage{bookmark}}{\usepackage{hyperref}}
\IfFileExists{xurl.sty}{\usepackage{xurl}}{} % add URL line breaks if available
\urlstyle{same}
\hypersetup{
  hidelinks,
  pdfcreator={LaTeX via pandoc}
}

% Custom header and footer
\pagestyle{fancy}
\fancyhead[l]{Azal Amer, Thalia Merkel, Austin Merkel}
\fancyhead[c]{Quantum 1 HW \#1}
\fancyhead[r]{\today}
\fancyfoot[c]{\thepage}
\renewcommand{\headrulewidth}{.2pt}
\setlength{\headheight}{15pt}

\begin{document}
\definecolor{shadecolor}{rgb}{0.9, 0.9, 0.9}

\hypertarget{problem-1}{%
\section{Problem 1}\label{problem-1}}

Given \[
\begin{align}
S_{x} = \frac{\hbar}{2}\left( \ket{+} \bra{-}+\ket{-} \bra{+}   \right)  \\
S_{y} = \frac{i\hbar}{2}\left(- \ket{x} \bra{-} +\ket{-} \bra{+}  \right)  \\
S_{z} = \frac{\hbar}{2}\left( \ket{+} \bra{+} -\ket{-} \bra{-}   \right) 
\end{align}
\]

\^{}states For readability sake, we're making the following notation
swap \[
\begin{align}
\ket{+} \to \ket{x}  \\
\ket{-}  \to \ket{\eta} 
\end{align}
\] There are 6 cases

\hypertarget{first-combo-s_xs_y}{%
\subsection{\texorpdfstring{First Combo
\([S_{x},S_{y}]\)}{First Combo {[}S\_\{x\},S\_\{y\}{]}}}\label{first-combo-s_xs_y}}

Starting with \(\left[ S_{x},S_{y} \right]\) \[
S_{x}S_{y} = \frac{i\hbar^{2}}{4}\left( \ket{x} \bra{\eta} +\ket{\eta} \bra{x}  \right) \left( -\ket{x} \bra{\eta} +\ket{\eta} \bra{x}  \right) 
\] We know that for orthogonal vectors, which \(\ket{x},\ket{\eta}\) are
defined to be, their inner product evaluates to zero. Additionally, the
inner product of a normalized vector with itself, is 1. Using that fact,
we reduce the above expression to \[
\begin{align}
S_{x}S_{y} = \frac{i\hbar^{2}}{4}\left[ \ket{x} \bra{x} -\ket{\eta} \bra{\eta}  \right]  \\
S_{x}S_{y} = \frac{i\hbar}{2}S_{z}
\end{align}
\] Now to calculate \(S_{y}S_{x}\) \[
\begin{align}
S_{y}S_{x} = \frac{i\hbar^2}{4} \left( -\ket{x} \bra{\eta} +\ket{\eta} \bra{x}  \right) \left( \ket{x} \bra{\eta} +\ket{\eta} \bra{x}  \right)  \\
= -\frac{i\hbar}{2}S_{z}
\end{align}
\] To get \(\left[ S_{z},S_{y} \right]\), you take the difference \[
\left[ S_{x},S_{y} \right] = S_{x}S_{y}-S_{y}S_{x} = \frac{2i\hbar}{2}S_{z} = i\hbar S_{z}
\] For \(\left\{ S_{x},S_{y} \right\}\) \[
\left\{ S_{x},S_{y} \right\}  = S_{x}S_{y} +S_{y}S_{z} = 0
\]

\hypertarget{second-combo-left-s_ys_z-right}{%
\subsection{\texorpdfstring{Second Combo
\(\left[ S_{y},S_{z} \right]\)}{Second Combo \textbackslash left{[} S\_\{y\},S\_\{z\} \textbackslash right{]}}}\label{second-combo-left-s_ys_z-right}}

\[
S_{y}S_{z} = \frac{i\hbar^{2}}{4}\left( -\ket{x} \bra{\eta} +\ket{\eta} \bra{x}  \right) \left( \ket{x} \bra{x} -\ket{\eta} \bra{\eta}  \right) 
\] Using the product operations from before this reduces to \[
\begin{align}
=\frac{i\hbar^{2}}{4} \left( \ket{x} \bra{\eta} +\ket{\eta} \bra{x}  \right)   \\
S_{y}S_{z}  =\frac{i\hbar}{2}S_{x}
\end{align}
\]

The other combo \[
\begin{align}
S_{z}S_{y} = \frac{i\hbar^2}{4}\left( \ket{x} \bra{x} -\ket{\eta} \bra{\eta}   \right)\left( -\ket{x} \bra{\eta} +\ket{\eta} \bra{x}  \right) \\
=\frac{i\hbar^{2}}{4}\left( -\ket{x} \bra{\eta} -\ket{\eta}\bra{x}   \right)  \\
S_{z}S_{y} = \frac{-i\hbar}{2}S_{x}
\end{align}
\] For the commutator \[
\left[ S_{z},S_{y} \right]  = S_{z}S_{y}-S_{y}S_{z} = i\hbar S_{z}
\] And for the anti-commutator \[
\left\{ S_{z},S_{y} \right\}  = S_{z}S_{y}iS_{y}S_{z} = 0
\]

\hypertarget{third-combo-left-s_xs_z-right}{%
\subsection{\texorpdfstring{Third Combo
\(\left[ S_{x},S_{z} \right]\)}{Third Combo \textbackslash left{[} S\_\{x\},S\_\{z\} \textbackslash right{]}}}\label{third-combo-left-s_xs_z-right}}

\[
\begin{align}
S_{x}S_{z} = \frac{\hbar^{2}}{4}\left( \ket{x}   \bra{\eta} +\ket{\eta} \bra{x} \right) \left( \ket{x} \bra{x} -\ket{\eta} \bra{\eta}  \right)  \\
=\frac{-i^2\hbar^{2}}{4}\left( -\ket{x} \bra{\eta} +\ket{\eta} \bra{x}  \right)  \\
S_{x}S_{z}= \frac{-i\hbar}{2}S_{y}
\end{align}
\]

\[
\begin{align}
S_{z}S_{x} = \frac{\hbar^{2}}{4}\left( \ket{x} \bra{x} -\ket{\eta} \bra{\eta}  \right)  \left( \ket{x}   \bra{\eta} +\ket{\eta} \bra{x} \right) \\
=\frac{-i^2\hbar^{2}}{4}\left( \ket{x} \bra{\eta} -\ket{\eta} \bra{x}  \right)  \\
S_{z}S_{x}= \frac{i\hbar}{2}S_{y}
\end{align}
\]

Given both of these,

\[
\begin{align}
\left[ S_{x},S_{z} \right]  = S_{x}S_{z} - S_{z}S_{x} = -i\hbar S_{y} \\
\left\{ S_{x},S_{z} \right\}  =  S_{x}S_{z} + S_{z}S_{x} = 0
\end{align}
\]

\hypertarget{commutator-symmetric-cases}{%
\subsection{Commutator Symmetric
Cases}\label{commutator-symmetric-cases}}

There are also the symmetric cases to consider \[
\begin{align}
\left[ S_{x} ,S_{x}\right]  = S_{x}^{2}-S_{x}^{2}=0 \\
\left[ S_{y} ,S_{y}\right]  = S_{y}^{2}-S_{y}^{2}=0 \\
\left[ S_{z} ,S_{z}\right]  = S_{z}^{2}-S_{z}^{2}=0
\end{align}
\] In conclusion, we have the 6 total below cases \[
\begin{align}
\left[ S_{x} ,S_{x}\right]  = S_{x}^{2}-S_{x}^{2}=0 \\
\left[ S_{y} ,S_{y}\right]  = S_{y}^{2}-S_{y}^{2}=0 \\
\left[ S_{z} ,S_{z}\right]  = S_{z}^{2}-S_{z}^{2}=0 \\
\left[ S_{x},S_{z} \right]  =  i\hbar S_{y} \\ 
\left[ S_{z},S_{y} \right]  =  i\hbar S_{z} \\
\left[ S_{x},S_{y} \right] =  i\hbar S_{z}
\end{align}
\]

We notice that for a general case of
\([S_{i}, S_{j}]=S_{i}S_{j}-S_{j}S_{i}\) \[
[S_{j},S_{i}] = S_{j}S_{i}-S_{i}S_{j} = -(S_{i}S_{j}-S_{j}S_{i}) \]
Thus, \[[S_{i,} S_{j}] = -[S_{j,} S_{i}]\]

So for each of the 3 non symmetric cases we showed, their opposite
counterparts are related by a sign.

Thus, by observing each case we can see that they obey the relation \[
[S_{i}. S_{j}] = i\hbar\epsilon_{ijk}S_{k}
\]

\hypertarget{anticommutator-symmetric-cases}{%
\subsection{Anticommutator Symmetric
Cases}\label{anticommutator-symmetric-cases}}

For the cases of \(\{S_{x}, S_{x}\}, \{S_{y}, S_{y}\}\) ,and
\(\{S_{z}, S_{z}\}\) we see that

\[
\{S_{i}, S_{i}\} = S_{i}S_{i}+S_{i}S_{i}=2S_{i}^2
\]

We also know that for each basis of \(\ket{x}, \ket{\eta}\)

\[
\sum_{i}^n\ket{a_{i}}\bra{a_{i}} = \ket{x}\bra{x} +\ket{\eta}\bra{\eta}=1       
\]

for \(i=x\):

\[
\begin{align}
2S_{x}^2=\frac{2\hbar^2}{4}(\ket{x}\bra{\eta}+\ket{\eta}\bra{x})(\ket{x}\bra{\eta}+\ket{\eta}\bra{x}) \\
\frac{2\hbar^2}{4}(\ket{x}\bra{x}+\ket{\eta}\bra{\eta}) \\
=\frac{\hbar^2}{2}
\end{align}
\]

for \(i=y\):

\[
\begin{align}
2S_{y}^2=\frac{2i^2\hbar^2}{4}(-\ket{x}\bra{\eta}+\ket{\eta}\bra{x})(-\ket{x}\bra{\eta}+\ket{\eta}\bra{x}) \\
\frac{-2\hbar^2}{4}(-\ket{x}\bra{x}-\ket{\eta}\bra{\eta}) \\
=\frac{\hbar^2}{2}
\end{align}
\]

for \(i=z\):

\[
\begin{align}
2S_{z}^2=\frac{2\hbar^2}{4}(\ket{x}\bra{x}-\ket{\eta}\bra{\eta})(\ket{x}\bra{x}-\ket{\eta}\bra{\eta}) \\
\frac{2\hbar^2}{4}(\ket{x}\bra{x}+\ket{\eta}\bra{\eta}) \\
=\frac{\hbar^2}{2}
\end{align}
\]

Thus, given these results and the ones above, we see that \[
\{S_{i}, S_{j}\} = \frac{\hbar^2}{2}\delta_{ij}
x\]

\hypertarget{problem-2}{%
\section{Problem 2}\label{problem-2}}

From office hours, we know that \(S\) is defined to be a vector who's
components are the Pauli Matrices \(\left( S_{x},S_{y},S_{z} \right)\).
To find the representative matrix of \(S \cdot \hat{r}\), we multiply
the components \[
S\cdot \hat{r} = S_{x}r_{x}+ S_{y}r_{y}+ S_{z}r_{z}
\] Then substituting in the values for the Pauli Matrices and r

\[
\begin{align}
= \frac{\hbar}{2}\left( \sin(\theta) \cos(\phi) \begin{pmatrix}
0 & 1 \\
1 & 0
\end{pmatrix}
+ \sin(\theta) \sin(\phi) \begin{pmatrix}
0 & -i \\
i & 0
\end{pmatrix}
+ \cos(\theta) \begin{pmatrix}
1 & 0 \\
0 & -1
\end{pmatrix} \right)  \\ 
=\left[\begin{matrix}\cos{\left(\theta \right)} & - i \sin{\left(\phi \right)} \sin{\left(\theta \right)} + \sin{\left(\theta \right)} \cos{\left(\phi \right)}\\i \sin{\left(\phi \right)} \sin{\left(\theta \right)} + \sin{\left(\theta \right)} \cos{\left(\phi \right)} & - \cos{\left(\theta \right)}\end{matrix}\right]
\end{align}
\]

This matrix represents a linear combination of our Pauli matrices, to
construct it in the \(\ket{\pm}\) basis, we find the eigenvectors with
eigenvalues. We can use Euler's formula on the antidiagonal terms to
reduce the complexity.

\[
\begin{align}
\pm i\sin(\phi)\sin(\theta)+\sin(\theta)\cos(\phi) \\
=\sin(\theta)(\cos (\phi)\pm i\sin(\phi)) \\
= e^{ \pm i\phi }\\
\end{align}
\] Plugging into the above matrix then gives us the below \[
S\cdot \hat{r}=\frac{\hbar}{2}\begin{bmatrix}\cos{\left(\theta \right)} & \sin(\theta)e^{ -i\phi }\\\sin(\theta)e^{ i\phi } & - \cos{\left(\theta \right)}\end{bmatrix}
\] In our main case, we only need the eigenvector corresponding to
\(\lambda=\frac{\hbar}{2}\), as that is the eigenvalue of \(\ket{+}\).

\[
eigen_{\lambda=1}(S\cdot \hat{r}) =\ket{v} =\begin{pmatrix}
\sin (\theta) e^{ -i\phi } \\
\cos( \theta)+1
\end{pmatrix}
\]

This above is then in the \(\ket{\pm}\) basis, as the Pauli observables
were constructed as such. However, the eigenvectors don't have an
amplitude attached, and need to be normalized to preserve the
probability summing to 1.

The magnitude of the eigenvector is just \[
\sqrt{\left<v|v\right>  } = \sqrt{ aa^{*} }
\] which after calculating, gives us a normalization constant of

\[
c=\frac{1}{\left( 2\cos (\theta)+2 \right) }
\]

This gives us a final representation of

\[
\ket{S\cdot \hat{r};+} =\frac{1}{\left( 2\cos (\theta)+2 \right) }\left( \sin(\theta)e^{ -i\phi }\ket{+} +(\cos \theta+1)\ket{-}  \right) 
\]

\hypertarget{problem-3}{%
\section{Problem 3}\label{problem-3}}

\[
H = a\left(\ket{1}\bra{1} -\ket{2}\bra{2}+\ket{2}\bra{1}+\ket{1}\bra{2}  \right) 
\]

We will write the hamiltonian in the \(\ket{1,2}\) basis, with the
following definitions \[
\begin{align}
\ket{1}  = \begin{pmatrix}
1 \\
0  
\end{pmatrix} \\
\ket{2} =\begin{pmatrix}
0 \\
1
\end{pmatrix}
\end{align}
\] These definitions then let us rewrite our definition of \(H\) as \[
\begin{align}
H = a\left( \begin{pmatrix}
1 \\ 0
\end{pmatrix}\begin{pmatrix}
1 &
0
\end{pmatrix}-\begin{pmatrix}
0 \\ 1
\end{pmatrix}\begin{pmatrix}
0 &
1
\end{pmatrix} +\begin{pmatrix}
0 \\ 1
\end{pmatrix}\begin{pmatrix}
1 &
0
\end{pmatrix}+\begin{pmatrix}
1 \\
0
\end{pmatrix}\begin{pmatrix}
0 &
1
\end{pmatrix}\right)  \\
H =a\begin{pmatrix}
1 & 0 \\
0  & 0
\end{pmatrix}-\begin{pmatrix}
0 & 0 \\
0 & 1
\end{pmatrix}+\begin{pmatrix}
0 & 0 \\
1 & 0
\end{pmatrix}+\begin{pmatrix}
0 & 1 \\
0  & 0
\end{pmatrix} \\
H = a\begin{pmatrix}
1 & 1 \\
1 & -1
\end{pmatrix}
\end{align}
\] After solving the eigenvector and eigenvalue problem for this matrix,
we get \[
\lambda
\]

\begin{Shaded}
\begin{Highlighting}[]
\NormalTok{import sympy as sp}
\NormalTok{a = sp.symbols(\textquotesingle{}a\textquotesingle{})}
\NormalTok{vectors = (a*sp.Matrix([}
\NormalTok{     [1,1],}
\NormalTok{     [1,{-}1]}
\NormalTok{])).eigenvects()}
\NormalTok{for i in range(len(vectors)):}
\NormalTok{    value =(vectors[i][2][0].T*vectors[i][2][0]).expand()}
\NormalTok{    print(sp.latex(sp.simplify((1/value[0])*vectors[i][2][0])))}
\end{Highlighting}
\end{Shaded}

\[
\left[ \left( - \sqrt{2} a, \ 1, \ \left[ \left[\begin{matrix}1 - \sqrt{2}\\1\end{matrix}\right]\right]\right), \ \left( \sqrt{2} a, \ 1, \ \left[ \left[\begin{matrix}1 + \sqrt{2}\\1\end{matrix}\right]\right]\right)\right]
\] \[
\begin{align}
\lambda_{1} =-a\sqrt{ 2 } &&\vec{v_{1}} = \begin{pmatrix}1 - \sqrt{2}\\1\end{pmatrix} \\
\lambda_{2} =a\sqrt{ 2 } &&\vec{v_{2}} = \begin{pmatrix}1 + \sqrt{2}\\1\end{pmatrix}
\end{align}
\]

These eigenvectors need to be normalized however. To calculate the
normalization constant of these eigenvectors, we divide each by their
inner product. \[
\begin{align}
\left<v_{1}|v_{1}\right> = \left[\begin{matrix}4 - 2 \sqrt{2}\end{matrix}\right]  \\
\left<v_{2}|v_{2}\right> =  \left[\begin{matrix}2 \sqrt{2} + 4\end{matrix}\right]
\end{align}
\] Then dividing off this value gives us the eigenkets below to
accompany our eigenvalues. \[
\begin{align}
\lambda_{1} =-a\sqrt{ 2 } &&\vec{v_{1}} = \frac{1}{\sqrt{ 4-2\sqrt{ 2 } }}\begin{pmatrix}1 - \sqrt{2}\\1\end{pmatrix} \\
\lambda_{2} =a\sqrt{ 2 } &&\vec{v_{2}} = \frac{1}{\sqrt{ 2\sqrt{ 2 }+4 }}\begin{pmatrix}1 + \sqrt{2}\\1\end{pmatrix}
\end{align}
\]

\hypertarget{problem-4}{%
\section{Problem 4}\label{problem-4}}

Find the eigenvalues and eigenvectors of the below matrix \[
\begin{pmatrix}
0 & -i & 0 & 0  \\
i & 0 & 0 & 0 \\
0 & 0 & 0 & 1 \\
0 & 0 & 1 & 0
\end{pmatrix}
\] \[
\begin{align}
\begin{pmatrix}
0 & -i & 0 & 0  \\
i & 0 & 0 & 0 \\
0 & 0 & 0 & 1 \\
0 & 0 & 1 & 0
\end{pmatrix}-\lambda I \\
=\begin{pmatrix}
-\lambda & -i & 0 & 0  \\
i & -\lambda & 0 & 0 \\
0 & 0 & -\lambda & 1 \\
0 & 0 & 1 & -\lambda
\end{pmatrix}
\end{align}
\] Working out the determinant of this matrix gives us \[
\det(M) = \left(\lambda^{2} - 1\right)^{2}
\] We can see that there are two degeneracies from here, two
eigenvectors with \(\lambda=1\), two with \(\lambda=-1\)

\begin{Shaded}
\begin{Highlighting}[]
\NormalTok{import sympy as sp}
\NormalTok{from sympy import Matrix, Symbol, I}

\NormalTok{\# Define the symbol for lambda}
\NormalTok{lambda\_sym = Symbol(\textquotesingle{}lambda\textquotesingle{})}

\NormalTok{\# Create the matrix}
\NormalTok{matrix = Matrix([}
\NormalTok{    [{-}lambda\_sym, {-}I, 0, 0],}
\NormalTok{    [I, {-}lambda\_sym, 0, 0],}
\NormalTok{    [0, 0, {-}lambda\_sym, 1],}
\NormalTok{    [0, 0, 1, {-}lambda\_sym]}
\NormalTok{])}

\NormalTok{print(matrix)}
\NormalTok{print(sp.latex(matrix.eigenvects()))}
\end{Highlighting}
\end{Shaded}

Solving the eigenvector problem is then just a matter of finding vector
solutions to \[
(M-\lambda I)\vec{v}=0
\] This then gives \[
\left[ \left( 1 - \lambda, \ 2, \ \left[ \left[\begin{matrix}- i\\1\\0\\0\end{matrix}\right], \ \left[\begin{matrix}0\\0\\1\\1\end{matrix}\right]\right]\right), \ \left( - \lambda - 1, \ 2, \ \left[ \left[\begin{matrix}i\\1\\0\\0\end{matrix}\right], \ \left[\begin{matrix}0\\0\\-1\\1\end{matrix}\right]\right]\right)\right]
\] \[
\begin{align}
\lambda=1:\begin{bmatrix}- i\\1\\0\\0\end{bmatrix},\begin{bmatrix}0\\0\\1\\1\end{bmatrix} \\ \\
\lambda=-1\begin{bmatrix}i\\1\\0\\0\end{bmatrix}, \ \begin{bmatrix}0\\0\\-1\\1\end{bmatrix}
\end{align}
\]
\end{document}