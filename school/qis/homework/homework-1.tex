\PassOptionsToPackage{unicode}{hyperref}
\PassOptionsToPackage{hyphens}{url}

\documentclass[]{article}

% Add the geometry package and set smaller margins
\usepackage[margin=1in]{geometry}
\usepackage{braket}

\usepackage{amsmath,amssymb}
\usepackage{lmodern}
\usepackage{iftex}
\ifPDFTeX
  \usepackage[T1]{fontenc}
  \usepackage[utf8]{inputenc}
  \usepackage{textcomp} % provide euro and other symbols
\else % if luatex or xetex
  \usepackage{unicode-math} % this also loads fontspec
  \defaultfontfeatures{Scale=MatchLowercase}
  \defaultfontfeatures[\rmfamily]{Ligatures=TeX,Scale=1}
\fi

\usepackage{fancyhdr}

% Use upquote if available, for straight quotes in verbatim environments
\IfFileExists{upquote.sty}{\usepackage{upquote}}{}

\IfFileExists{microtype.sty}{
  \usepackage[]{microtype}
  \UseMicrotypeSet[protrusion]{basicmath} % disable protrusion for tt fonts
}{}

\makeatletter
\@ifundefined{KOMAClassName}{% if non-KOMA class
  \IfFileExists{parskip.sty}{
    \usepackage{parskip}
  }{% else
    \setlength{\parindent}{0pt}
    \setlength{\parskip}{6pt plus 2pt minus 1pt}
  }
}{% if KOMA class
  \KOMAoptions{parskip=half}
}
\makeatother

\usepackage{xcolor}
\usepackage{longtable,booktabs,array}
\usepackage{calc} % for calculating minipage widths

% Correct order of tables after \paragraph or \subparagraph
\usepackage{etoolbox}
\makeatletter
\patchcmd\longtable{\par}{\if@noskipsec\mbox{}\fi\par}{}{}
\makeatother

% Allow footnotes in longtable head/foot
\IfFileExists{footnotehyper.sty}{\usepackage{footnotehyper}}{\usepackage{footnote}}
\makesavenoteenv{longtable}

\setlength{\emergencystretch}{3em} % prevent overfull lines
\providecommand{\tightlist}{%
  \setlength{\itemsep}{0pt}\setlength{\parskip}{0pt}}
\setcounter{secnumdepth}{-\maxdimen} % remove section numbering

\ifLuaTeX
  \usepackage{selnolig} % disable illegal ligatures
\fi

\IfFileExists{bookmark.sty}{\usepackage{bookmark}}{\usepackage{hyperref}}
\IfFileExists{xurl.sty}{\usepackage{xurl}}{} % add URL line breaks if available
\urlstyle{same}
\hypersetup{
  hidelinks,
  pdfcreator={LaTeX via pandoc}
}

% Custom header and footer
\pagestyle{fancy}
\fancyhead[l]{Azal Amer}
\fancyhead[c]{QIS HW \#1}
\fancyhead[r]{\today}
\fancyfoot[c]{\thepage}
\renewcommand{\headrulewidth}{.2pt}
\setlength{\headheight}{15pt}

\begin{document}
\hypertarget{problem-1}{%
\section{Problem 1}\label{problem-1}}

\hypertarget{a}{%
\subsection{(a)}\label{a}}

\begin{longtable}[]{@{}ll@{}}
\toprule\noalign{}
Stochastic & Unitary \\
\midrule\noalign{}
\endhead
\bottomrule\noalign{}
\endlastfoot
B & B \\
C & D \\
& F \\
& G \\
& \\
\end{longtable}

\hypertarget{b}{%
\subsection{\texorpdfstring{\textbf{(b)}}{(b)}}\label{b}}

Show that any matrix which is stochastic and unitary, must be a
permutation matrix

For any Unitary matrix, we know that \[
UU^{\dagger}=I
\] From the definition of a stochastic matrix, we know that all entries
are real, and non-negative. From this we know that all entries in \(U\)
are real. \[
U^{T} = U^{\dagger}
\] Rewriting our definition of Identity, we then get \[
UU^{T} = I
\] Given, \(i,j,j'<N\) where N is the dimension of our matrix.

When \(j\neq j'\). Assuming \[
P_{i,j},P_{i,j'}> 0
\] This then tells us \[
{PP^{T}}_{jj'}=I_{jj'}> 0
\] because the product of two numbers greater than zero, is greater than
zero. However this is a contradiction, entries not along the diagonal of
the identity must be zero.

Therefore, for any column of the matrix, this means for any \(j,j'\) in
a row, the product of any two must be zero. However because we are in a
stochastic matrix, the sum of the columns must be one. Therefore, we
must have a permutation matrix

\hypertarget{c}{%
\subsection{(c)}\label{c}}

\[
\begin{pmatrix}
0 & 1 \\
1 & 0
\end{pmatrix}
\]

\hypertarget{d}{%
\subsection{(d)}\label{d}}

\[
\begin{pmatrix}
\cos^{2} & -\sin^{2} \\
\sin^{2} & \cos^{2}\theta
\end{pmatrix}
\] I know that any permutation of identity should work through basic
deductive reasoning. The norm is even, so we should be covered on
negative numbers

\hypertarget{problem-2}{%
\section{Problem 2}\label{problem-2}}

\hypertarget{a-1}{%
\subsection{(a)}\label{a-1}}

Calculate the tensor product

\[
\begin{bmatrix}
\frac{2}{3} \\
\frac{1}{3}
\end{bmatrix}\otimes \begin{bmatrix}
\frac{1}{5} \\
\frac{4}{5}
\end{bmatrix}
\] Calculating the tensor product means putting the second vector into
the first \[
\begin{align}
\begin{pmatrix}
\frac{2}{3}\begin{pmatrix}
\frac{1}{5} \\
\frac{4}{5}
\end{pmatrix} \\
\frac{1}{3}\begin{pmatrix}
\frac{1}{5} \\
\frac{4}{5}
\end{pmatrix}
\end{pmatrix}=\begin{pmatrix}
\frac{2}{15} \\
\frac{8}{15} \\
\frac{1}{15} \\
\frac{4}{15}
\end{pmatrix}
\end{align}
\]

\hypertarget{b-1}{%
\subsection{(b)}\label{b-1}}

\hypertarget{a-2}{%
\subsubsection{A:}\label{a-2}}

\[
\begin{bmatrix}
\frac{2}{9 } \\
\frac{1}{9} \\
\frac{4}{9} \\
\frac{2}{9}
\end{bmatrix}=\begin{pmatrix}
\frac{2}{9} \\
\frac{1}{9} \\
\end{pmatrix}\otimes \begin{pmatrix}
1 \\
2
\end{pmatrix}
\]

\hypertarget{b-2}{%
\subsubsection{B:}\label{b-2}}

\[
\begin{bmatrix}
0 \\
1 \\
0 \\
0
\end{bmatrix}=\begin{pmatrix}
1 \\
0
\end{pmatrix}\otimes \begin{pmatrix}
0 \\
1
\end{pmatrix}
\]

\hypertarget{c-1}{%
\subsubsection{C:}\label{c-1}}

\[
\begin{bmatrix}
\frac{1}{4} \\
\frac{1}{4} \\
\frac{1}{4} \\
\frac{1}{4}
\end{bmatrix}=\begin{pmatrix}
\frac{1}{2} \\
\frac{1}{2}
\end{pmatrix}\otimes \begin{pmatrix}
\frac{1}{2} \\
\frac{1}{2} \\
\end{pmatrix}
\]

\hypertarget{d-1}{%
\subsubsection{D:}\label{d-1}}

\[
\begin{bmatrix} 
0 \\
\frac{1}{2} \\
0 \\
\frac{1}{2}
\end{bmatrix}=\begin{pmatrix}
0 \\
1
\end{pmatrix}\otimes \begin{pmatrix}
\frac{1}{2} \\
\frac{1}{2}
\end{pmatrix}
\]

\hypertarget{c-2}{%
\subsection{(c)}\label{c-2}}

Prove that there is no \(2x2\) real matrix such that \[
A^{2} = \begin{bmatrix}
1 & 0 \\
0 & -1
\end{bmatrix}
\]

Firstly, note that the matrix \(A^{2}\) is diagonalized. This means that
\(\left\{ 1,-1 \right\}\) is the set of it's eigenvalues. Any matrix B
can be factored in terms of it's eigenvectors and eigenvalues like so \[
B = PDP^{-1}
\] Where \(P\) is a matrix constructed by the eigenvectors and \(D\) is
constructed by the eigenvalues along the eigenvectors. The square-root
of a diagonal matrix has eigenvalues analogous to the square root of
it's eigenvalues. That means that \(A\) has eigenvalues
\(1,\sqrt{ -1 }\). \textbf{From here, we need to show no real matrix can
have complex eigenvalues if diagonal}. \[
\ket{\psi}  = e^{ i\theta }\ket{\psi} 
\] Both are the same, it's just a matter of rotation inside the Hilbert
Space.

\hypertarget{problem-3}{%
\section{Problem 3}\label{problem-3}}

\hypertarget{a-3}{%
\subsection{(a)}\label{a-3}}

Let \(\ket{\psi}=\frac{{\ket{0}+2\ket{1}}}{\sqrt{ 5 }}\) and
\(\ket{\phi} = \frac{{2i\ket{0}+3\ket{1}}}{\sqrt{ 13 }}\). What's
\(\left<\psi|\phi\right>\)

\[
\left<\psi|\phi\right> = \ket{\psi} ^{\dagger}\ket{\phi} 
\]

Now we need to calculate the hermitian of \(\ket{\psi}\), which is the
conjugate transpose. In this case, \(\ket{\psi}\in\mathbb{R}\), so we
just need the transpose

\[
\ket{\psi}  = \frac{1}{\sqrt{ 5 }}\begin{bmatrix}
1 \\
2
\end{bmatrix}=\frac{1}{\sqrt{ 5 }}\begin{bmatrix}
1 & 2
\end{bmatrix}
\]

Now we multiply it with \(\ket{\phi}\) using matrix math/

\[
\begin{align}
\left<\psi|\phi\right> = \frac{1}{\sqrt{ 5 }}\begin{bmatrix}
1 & 2
\end{bmatrix}* \frac{1}{\sqrt{ 13 }}\begin{bmatrix}
2i \\
3
\end{bmatrix} \\
\left<\psi|\phi\right> =\frac{1}{\sqrt{ 65 }}\left( 2i*1+2*3 \right) =\frac{2i+6}{\sqrt{ 65 }}
\end{align}
\]

\hypertarget{b-3}{%
\subsection{(b)}\label{b-3}}

Normalize \(\ket{\phi} = 2i\ket{0}-3i\ket{1}\) To normalize the vector,
we want to make sure the sum of their probabilities squared, adds to
\(1\). \[
A = \sqrt{ (2i)(-2i)+(3i)(-3i) } = \sqrt{ 4+9 } = \sqrt{ 13 }
\]

\hypertarget{c-3}{%
\subsection{(c)}\label{c-3}}

To show that every vector can be composed of those vectors, we just have
to show that they form an orthonormal basis {[}{[}../../Quantum
Computing/Definitions/Orthonormal Bases\#\^{}506716\textbar{} {]}{]}.
First, they have to have an inner-product of zero. Then they both need
to have a magnitude of 1. If both vectors can form a unitary, then they
span \(\mathbb{C}\) First aiming at the beginning condition. \[
\begin{align}
\ket{i} =\frac{1}{\sqrt{ 2 }}\left( \ket{0} +i\ket{1}  \right) \\
\ket{-i} = \frac{1}{\sqrt{ 2 }}(\ket{0} -i\ket{1} )  
\end{align}
\]

The inner product as an operation is commutative, which means \[
|\left<\psi|\phi\right>| = |\left<\phi|\psi\right>  |
\] so we only need to check one of the two possible cases to prove
orthogonality. \[
\ket{i}  = \frac{1}{\sqrt{ 2 }} \begin{bmatrix}
1 \\
i
\end{bmatrix}
\] In this case, we're going to check \[
\begin{align}
\left<i|-i \right> \\
\braket{ i}  = \ket{i} ^{\dagger} \\
\ket{i} ^{\dagger} = \frac{1}{\sqrt{ 2 }}\begin{bmatrix}
1 & -i
\end{bmatrix}=\bra{i}  \\  
\left<i |  -i\right>  = \frac{1}{2} \begin{pmatrix}
1 & -i
\end{pmatrix}\begin{pmatrix}
1 \\
-i
\end{pmatrix} \\
\left<i|-i\right> = \frac{1}{2}\left( 1^{2}+(-i)^{2} \right) \\
\left<i|-i\right>   =0
\end{align}
\]

The vectors are orthogonal, but now we need to check if the vectors are
normalized already. This just means
\(\ket{i}\ket{i}^{*},\ket{-i}\ket{-i}^{*}=1\) If the sum of the square
of amplitudes (Borns Rule), is 1, then we're golden. \[
\begin{align}
P_{\ket{i} } =\left( \frac{1}{2}(1*1+(i*-i)) \right)  = 1 \\
P_{\ket{-i} } =\left( \frac{1}{2}(1*1+(-i*i)) \right)  = 1 \\
\end{align}
\] Both probability sums are 1, and both vectors are orthogonal, which
means that the two vectors can form an orthonormal basis over
\(\mathbb{C}^{2}\)

\hypertarget{d-2}{%
\subsection{(d)}\label{d-2}}

We want to find the operator which transforms us from the
\(\left\{ \ket{0},\ket{1} \right\}\) basis, to the
\(\left\{ \ket{i},\ket{-i} \right\}\) basis. We can find this unitary
through \[
U = \ket{i} \bra{i} -\ket{-i} \bra{-i}  
\] as the corresponding unitary will have those vectors as it's
eigenvectors. The corresponding eigenvalues are \(\lambda=\pm 1\).

Calculating \(U\), we need \[
\begin{align}
\ket{i} = \frac{1}{\sqrt{ 2 }}\begin{bmatrix}
1 \\
i 
\end{bmatrix} \\
\ket{-i} = \frac{1}{\sqrt{ 2 }} \begin{bmatrix}
1 \\
-i
\end{bmatrix} \\
\bra{i}  = \begin{pmatrix}
1 & -i
\end{pmatrix} \\
\bra{-i} = \begin{pmatrix}
1 & i
\end{pmatrix} 
\end{align}
\] Now we just do our outer products. Starting with \(\ket{i}\) \[
\begin{align}
\ket{i} \bra{i} -\ket{-i} \bra{-i}   = \frac{1}{2}\begin{bmatrix}
1 \\
i
\end{bmatrix}\begin{bmatrix}
1 & -i
\end{bmatrix}-\frac{1}{2}\begin{bmatrix}
1 \\
-i 
\end{bmatrix}\begin{pmatrix}
1 & i
\end{pmatrix} \\
= \frac{1}{2}\begin{pmatrix}
1 & -i  \\
i & 1
\end{pmatrix}-\frac{1}{2}\begin{pmatrix}
1 & i \\
-i & 1
\end{pmatrix} \\
=\frac{1}{2}\begin{pmatrix}
0 & -2i \\
2i & 0
\end{pmatrix} \\
U=\begin{pmatrix}
0 & -i \\
i & 0
\end{pmatrix}
\end{align}
\]

After finding this unitary \(U\), we just transform our original state
\[
U\ket{\psi}  = \ket{\psi'} 
\] \emph{Note both vectors are identical, but are denoted slightly
differently as one is in a separate basis} \[
\begin{align}
\ket{\psi'}  = U\ket{\psi} \\
=\begin{pmatrix}
0 & -i \\
i & 0
\end{pmatrix}* \frac{1}{\sqrt{ 5 }}\begin{bmatrix}
1 \\
2
\end{bmatrix} \\
= \frac{1}{\sqrt{ 5 }}\begin{pmatrix}
-2i \\
i
\end{pmatrix}
\end{align}
\] Therefore \(\ket{\psi}\) in the \(\left\{ \ket{i},\ket{-i} \right\}\)
basis is \[
\ket{\psi}  = \frac{1}{\sqrt{ 5 }}\left( -2i\ket{i} +i\ket{-i}  \right) 
\]
\end{document}