\PassOptionsToPackage{unicode}{hyperref}
\PassOptionsToPackage{hyphens}{url}

\documentclass[]{article}


\usepackage{color}
\usepackage{graphicx}

\usepackage{fancyvrb}
\newcommand{\VerbBar}{|}
\newcommand{\VERB}{\Verb[commandchars=\\\{\}]}
\DefineVerbatimEnvironment{Highlighting}{Verbatim}{commandchars=\\\{\}}
% Add ',fontsize=\small' for more characters per line
\usepackage{framed}
\definecolor{shadecolor}{RGB}{248,248,248}
\newenvironment{Shaded}{\begin{snugshade}}{\end{snugshade}}
\newcommand{\AlertTok}[1]{\textcolor[rgb]{0.94,0.16,0.16}{#1}}
\newcommand{\AnnotationTok}[1]{\textcolor[rgb]{0.56,0.35,0.01}{\textbf{\textit{#1}}}}
\newcommand{\AttributeTok}[1]{\textcolor[rgb]{0.13,0.29,0.53}{#1}}
\newcommand{\BaseNTok}[1]{\textcolor[rgb]{0.00,0.00,0.81}{#1}}
\newcommand{\BuiltInTok}[1]{#1}
\newcommand{\CharTok}[1]{\textcolor[rgb]{0.31,0.60,0.02}{#1}}
\newcommand{\CommentTok}[1]{\textcolor[rgb]{0.56,0.35,0.01}{\textit{#1}}}
\newcommand{\CommentVarTok}[1]{\textcolor[rgb]{0.56,0.35,0.01}{\textbf{\textit{#1}}}}
\newcommand{\ConstantTok}[1]{\textcolor[rgb]{0.56,0.35,0.01}{#1}}
\newcommand{\ControlFlowTok}[1]{\textcolor[rgb]{0.13,0.29,0.53}{\textbf{#1}}}
\newcommand{\DataTypeTok}[1]{\textcolor[rgb]{0.13,0.29,0.53}{#1}}
\newcommand{\DecValTok}[1]{\textcolor[rgb]{0.00,0.00,0.81}{#1}}
\newcommand{\DocumentationTok}[1]{\textcolor[rgb]{0.56,0.35,0.01}{\textbf{\textit{#1}}}}
\newcommand{\ErrorTok}[1]{\textcolor[rgb]{0.64,0.00,0.00}{\textbf{#1}}}
\newcommand{\ExtensionTok}[1]{#1}
\newcommand{\FloatTok}[1]{\textcolor[rgb]{0.00,0.00,0.81}{#1}}
\newcommand{\FunctionTok}[1]{\textcolor[rgb]{0.13,0.29,0.53}{\textbf{#1}}}
\newcommand{\ImportTok}[1]{#1}
\newcommand{\InformationTok}[1]{\textcolor[rgb]{0.56,0.35,0.01}{\textbf{\textit{#1}}}}
\newcommand{\KeywordTok}[1]{\textcolor[rgb]{0.13,0.29,0.53}{\textbf{#1}}}
\newcommand{\NormalTok}[1]{#1}
\newcommand{\OperatorTok}[1]{\textcolor[rgb]{0.81,0.36,0.00}{\textbf{#1}}}
\newcommand{\OtherTok}[1]{\textcolor[rgb]{0.56,0.35,0.01}{#1}}
\newcommand{\PreprocessorTok}[1]{\textcolor[rgb]{0.56,0.35,0.01}{\textit{#1}}}
\newcommand{\RegionMarkerTok}[1]{#1}
\newcommand{\SpecialCharTok}[1]{\textcolor[rgb]{0.81,0.36,0.00}{\textbf{#1}}}
\newcommand{\SpecialStringTok}[1]{\textcolor[rgb]{0.31,0.60,0.02}{#1}}
\newcommand{\StringTok}[1]{\textcolor[rgb]{0.31,0.60,0.02}{#1}}
\newcommand{\VariableTok}[1]{\textcolor[rgb]{0.00,0.00,0.00}{#1}}
\newcommand{\VerbatimStringTok}[1]{\textcolor[rgb]{0.31,0.60,0.02}{#1}}
\newcommand{\WarningTok}[1]{\textcolor[rgb]{0.56,0.35,0.01}{\textbf{\textit{#1}}}}
% Add the geometry package and set smaller margins
\usepackage[margin=1in]{geometry}

\usepackage{amsmath,amssymb}
\usepackage{lmodern}
\usepackage{iftex}
\ifPDFTeX
  \usepackage[T1]{fontenc}
  \usepackage[utf8]{inputenc}
  \usepackage{textcomp} % provide euro and other symbols
\else % if luatex or xetex
  \usepackage{unicode-math} % this also loads fontspec
  \defaultfontfeatures{Scale=MatchLowercase}
  \defaultfontfeatures[\rmfamily]{Ligatures=TeX,Scale=1}
\fi
\usepackage{braket}
\usepackage{fancyhdr}
\usepackage{listings}
\usepackage{framed}
\lstnewenvironment{Highlighting}{}{}
\usepackage{xcolor}
% Use upquote if available, for straight quotes in verbatim environments
\IfFileExists{upquote.sty}{\usepackage{upquote}}{}

\IfFileExists{microtype.sty}{
  \usepackage[]{microtype}
  \UseMicrotypeSet[protrusion]{basicmath} % disable protrusion for tt fonts
}{}

\makeatletter
\@ifundefined{KOMAClassName}{% if non-KOMA class
  \IfFileExists{parskip.sty}{
    \usepackage{parskip}
  }{% else
    \setlength{\parindent}{0pt}
    \setlength{\parskip}{6pt plus 2pt minus 1pt}
  }
}{% if KOMA class
  \KOMAoptions{parskip=half}
}
\makeatother

\usepackage{xcolor}
\usepackage{longtable,booktabs,array}
\usepackage{calc} % for calculating minipage widths

% Correct order of tables after \paragraph or \subparagraph
\usepackage{etoolbox}
\makeatletter
\patchcmd\longtable{\par}{\if@noskipsec\mbox{}\fi\par}{}{}
\makeatother

% Allow footnotes in longtable head/foot
\IfFileExists{footnotehyper.sty}{\usepackage{footnotehyper}}{\usepackage{footnote}}
\makesavenoteenv{longtable}

\setlength{\emergencystretch}{3em} % prevent overfull lines
\providecommand{\tightlist}{%
  \setlength{\itemsep}{0pt}\setlength{\parskip}{0pt}}
\setcounter{secnumdepth}{-\maxdimen} % remove section numbering

\usepackage{framed}
\usepackage{listings}

\ifLuaTeX
  \usepackage{selnolig} % disable illegal ligatures
\fi

\IfFileExists{bookmark.sty}{\usepackage{bookmark}}{\usepackage{hyperref}}
\IfFileExists{xurl.sty}{\usepackage{xurl}}{} % add URL line breaks if available
\urlstyle{same}
\hypersetup{
  hidelinks,
  pdfcreator={LaTeX via pandoc}
}

% Custom header and footer
\pagestyle{fancy}
\fancyhead[l]{Azal Amer}
\fancyhead[c]{QIS HW  \#3}
\fancyhead[r]{\today}
\fancyfoot[c]{\thepage}
\renewcommand{\headrulewidth}{.2pt}
\setlength{\headheight}{15pt}

\begin{document}
\definecolor{shadecolor}{rgb}{0.9, 0.9, 0.9}

\hypertarget{problem-1}{%
\section{Problem 1}\label{problem-1}}


Let's define the unitary matrix Alice uses in the form \[
U(\theta,\phi,\lambda)=
\begin{pmatrix}
\cos \frac{\theta}{2} & -e^{ i\lambda  }\sin \frac{\theta}{2} \\
e^{ i\phi  }\sin \frac{\theta}{2} & e^{ i(\lambda+\phi) }\cos{\frac{\theta}{2}}
\end{pmatrix}
\]

\begin{Shaded}
\begin{Highlighting}[]
\NormalTok{unitary = ltx2sp(r"""}
\NormalTok{\textbackslash{}begin\{pmatrix\}}
\NormalTok{\textbackslash{}cos \textbackslash{}frac\{\textbackslash{}theta\}\{2\} \& {-}e\^{}\{ i\textbackslash{}lambda  \}\textbackslash{}sin \textbackslash{}frac\{\textbackslash{}theta\}\{2\} \textbackslash{}\textbackslash{}}
\NormalTok{e\^{}\{ i\textbackslash{}phi  \}\textbackslash{}sin \textbackslash{}frac\{\textbackslash{}theta\}\{2\} \& e\^{}\{ i(\textbackslash{}lambda+\textbackslash{}phi) \}\textbackslash{}cos\{\textbackslash{}frac\{\textbackslash{}theta\}\{2\}\}}
\NormalTok{\textbackslash{}end\{pmatrix\}""")}

\NormalTok{ketH = sp.Matrix([1,0])}
\NormalTok{ketV = sp.Matrix([0,1])}
\NormalTok{print(sp.latex(unitary*ketV))}
\end{Highlighting}
\end{Shaded}

Applying the unitary to \(\ket{0},\ket{1}\) gives us \[
U\ket{0}  = \left[\begin{matrix}\cos{\left(\frac{\theta}{2} \right)}\\e^{i \phi} \sin{\left(\frac{\theta}{2} \right)}\end{matrix}\right]
\] \[
U\ket{1}  = \left[\begin{matrix}- e^{i \lambda} \sin{\left(\frac{\theta}{2} \right)}\\e^{i \left(\lambda + \phi\right)} \cos{\left(\frac{\theta}{2} \right)}\end{matrix}\right]
\] We then plug this back into the \(\ket{EPR}\) wave-function to get \[
\begin{align}
U_{Alice}\ket{EPR}  = \frac{1}{\sqrt{ 2 }} \left[ {\left( \cos \frac{\theta}{2}\ket{0} +e^{ i\phi }\sin \frac{\theta}{2}\ket{1}  \right)\otimes \ket{0} +\left(- e^{ -i\lambda }\sin \frac{\theta}{2}\ket{0} +e^{ i\left( \lambda+\phi \right) }\cos \frac{\theta}{2}\ket{1} \right)\otimes \ket{1}   } \right] \\
=\frac{1}{\sqrt{ 2 }}\left[ \cos \frac{\theta}{2}\ket{00}+e^{ i\phi }\sin \frac{\theta}{2}\ket{10} -e^{ i\lambda }\sin \frac{\theta}{2}\ket{01} +e^{ i(\lambda+\phi) }\cos \frac{\theta}{2}\ket{11}   \right] 
\end{align}
\]

We then define \(U^{T}\) as \[
U^{T} = 
\begin{pmatrix}
\cos \frac{\theta}{2} & e^{ i\phi  }\sin \frac{\theta}{2}\\-e^{ i\lambda  }\sin \frac{\theta}{2} 
 & e^{ i(\lambda+\phi) }\cos{\frac{\theta}{2}}
\end{pmatrix}
\] Then the transformed \(\left\{ \ket{0},\ket{1} \right\}\) states are
\[
\begin{align}
U^{T}\ket{0}  = \begin{bmatrix}
\cos \frac{\theta}{2} \\
-e^{ i\lambda }\sin \frac{\theta}{2} 
\end{bmatrix} \\
U^{T}\ket{1}  = \begin{bmatrix}
e^{ i\phi }\sin \frac{\theta}{2} \\
e^{ i(\lambda+\phi) }\cos \frac{\theta}{2}
\end{bmatrix}
\end{align}
\] Now plugging in that transformation to Bob's qubit gives us \[
\begin{align}
U^{T}_{Bob} \ket{EPR}  = \frac{1}{\sqrt{ 2 }}\left[  \ket{0} \otimes \left(\cos \frac{\theta}{2}\ket{0} - e^{ i\lambda }\sin \frac{\theta}{2}\ket{1}   \right)+\ket{1} \otimes \left( e^{ i\phi }\sin \frac{\theta}{2}\ket{0} +e^{ i(\lambda+\phi) }\cos \frac{\theta}{2}\ket{1}  \right)  \right]  \\
=\frac{1}{\sqrt{ 2 }}\left[ \cos \frac{\theta}{2}\ket{00}  -e^{ i\lambda }\sin \frac{\theta}{2}\ket{01} +e^{ i\phi }\sin \frac{\theta}{2}\ket{10} +e^{ i(\lambda+\phi) }\cos \frac{\theta}{2}\ket{11} \right] 
\end{align}
\] Therefore \[
U_{Alice}\ket{EPR}  = U_{Bob}^{T}\ket{EPR} 
\]


\newpage

\hypertarget{problem-2}{%
\section{Problem 2}\label{problem-2}}

\hypertarget{a}{%
\subsection{a)}\label{a}}

\begin{figure}
\centering
\includegraphics{../../../Supplemental Files/images/Pasted image 20240922190108.png}
\caption{Z circuit}
\end{figure}

For the above gate's, \[
\begin{align}
H = \frac{1}{\sqrt{ 2 }}\begin{pmatrix}
1 & 1  \\
1 & -1
\end{pmatrix} \\
X = \frac{1}{\sqrt{ 2 }}\begin{pmatrix}
0 & 1 \\
1 & 0
\end{pmatrix}
\end{align}
\] Which gives us an expression of \[
\begin{align}
Z =^{?} \frac{1}{\sqrt{ 2 }}\begin{pmatrix}
1 & 1  \\
1 & -1
\end{pmatrix}\begin{pmatrix}
0 & 1 \\
1 & 0
\end{pmatrix}\frac{1}{\sqrt{ 2 }}\begin{pmatrix}
1 & 1  \\
1 & -1
\end{pmatrix} \\
\frac{1}{2}\begin{pmatrix}
1 & 1 \\
-1 & 1
\end{pmatrix}\begin{pmatrix}
1 & 1 \\
1 & -1
\end{pmatrix} \\
=\frac{1}{2}\begin{pmatrix}
2 & 0 \\
0 & -2
\end{pmatrix} \\
=\begin{pmatrix}
1 & 0 \\
0 & -1
\end{pmatrix}
\end{align}
\] To then show

\begin{figure}
\centering
\includegraphics{../../../Supplemental Files/images/Pasted image 20240922190418.png}
\caption{X circuit}
\end{figure}

We'll just perform the same matrix multiplication steps here:

\[
\begin{align}
X =^{?} \frac{1}{\sqrt{ 2 }}\begin{pmatrix}
1 & 1  \\
1 & -1
\end{pmatrix}\begin{pmatrix}
1 & 0 \\
0 & -1
\end{pmatrix}\frac{1}{\sqrt{ 2 }}\begin{pmatrix}
1 & 1  \\
1 & -1
\end{pmatrix} \\
=\frac{1}{2}\begin{pmatrix}
1 & -1 \\
1 & 1
\end{pmatrix}\begin{pmatrix}
1 & 1 \\
1 & -1
\end{pmatrix} \\
\frac{1}{2}\begin{pmatrix}
0 & 2 \\
2 & 0
\end{pmatrix} \\
=\begin{pmatrix}
0 & 1 \\
1 & 0
\end{pmatrix}
\end{align}
\]

\hypertarget{b}{%
\subsection{b)}\label{b}}

It's \[
H_{0}C^{X}_{10}H_{0}= C^{Z}_{10}
\] These gate are equivalent, because as proven before, an \(X\) gate
sandwiched by Hadamards is equivalent to a \(Z\) gate. In this context,
the Hadamards flip the controlled result. If the condition is met to
flip the gate, the not gate is activated and the hadamards apply a
\(Z\). If the condition to flip isn't met, then the control applies an
identity. That is equivalent to \[
\begin{align}
HIH = HH \\
HH = \frac{1}{2}\begin{pmatrix}
1 & 1  \\
1 & -1
\end{pmatrix}\begin{pmatrix}
1 & 1  \\
1 & -1
\end{pmatrix} \\
=\begin{pmatrix}
1 & 0 \\
0 & 1
\end{pmatrix}
\end{align}
\] Which means that no trigger creates an identity.

\hypertarget{c}{%
\subsection{c)}\label{c}}

We know from an earlier proof, that two hadamards enclosing a not, make
a \(Z\). This gives us \[
H_{1}Csign_{1,2}H_{1}
\] We then know from the hint, that CSIGN is symmetric so this is equal
to \[
H_{1}CSIGN_{2,1}H_{1}
\] Two hadamards on a \(Z\) observable then make a \(X\), so this
becomes a \(CNOT_{2,1}\)

\newpage

\hypertarget{problem-3}{%
\section{Problem 3}\label{problem-3}}

\hypertarget{a-1}{%
\subsection{a)}\label{a-1}}

\[
\begin{align}
TH\ket{0} =\begin{bmatrix}
1 & 0 \\
0 & e^{ i\pi/4 }
\end{bmatrix} \frac{1}{\sqrt{ 2 }}\begin{bmatrix}
1 & 1 \\
1 & -1
\end{bmatrix}\begin{pmatrix}
1 \\
0
\end{pmatrix} \\
\frac{1}{\sqrt{ 2 }}\begin{bmatrix}
1 & 1 \\
e^{ i\pi/4 } & -e^{ i\pi/4 }
\end{bmatrix}\begin{pmatrix}
1 \\
0
\end{pmatrix} \\
=\frac{1}{\sqrt{ 2 }}\begin{pmatrix}
1 \\
e^{ i\pi/4 }
\end{pmatrix} \\
=\frac{1}{\sqrt{ 2 }}\left[ \ket{0} +e^{ i\pi/4 }\ket{1}  \right] 
\end{align}
\]

\hypertarget{b-1}{%
\subsection{b)}\label{b-1}}

Right before the control not, \(\ket{\psi_{in}}\) is \[
\ket{\psi_{in}}  =\alpha \ket{1} +\beta \ket{0} 
\] Applying the control-not gate, means first tensoring our
wave-functions together, then flipping the bits.

\[
H\ket{0} = \frac{1}{\sqrt{ 2 }}\ket{0} + \frac{1}{\sqrt{ 2 }}\ket{1}  
\] Then the phase gate \[
TH\ket{0} = \frac{1}{\sqrt{ 2 }}\ket{0} +\frac{e^{ i\pi/4 }}{\sqrt{ 2 }}\ket{1}  
\]

\[
\begin{align}
TH\ket{0} X\ket{\psi_{in}}  = \frac{1}{\sqrt{ 2 }}\left[ \ket{0} +e^{ i\pi/4 }\ket{1}  \right] \otimes \left[ \alpha \ket{1} +\beta \ket{0} \right]  \\
\frac{1}{\sqrt{ 2 }}\left[ \alpha \ket{01} +\alpha e^{ i\pi/4 }\ket{11} +\beta \ket{00} +\beta e^{ i\pi/4 }\ket{10}  \right] 
\end{align}
\] Now applying the control-not, controlling on \(\psi_{in}\), targeting
\(\ket{0}\).

\[
\begin{align}
\frac{1}{\sqrt{ 2 } }\left[ \alpha \ket{11} +\alpha e^{ i\pi/4 }\ket{01}+\beta \ket{00} +\beta e^{ i\pi/4 }\ket{10}   \right] \\
\end{align}
\] Then applying the \(P\) gate to the \(\ket{\psi}\) qubit, we know
that \[
\begin{align}
P\ket{0}  = \ket{0}  \\
P\ket{1}  = i\ket{1}  
\end{align}
\] Following that rule, we get

\[
\frac{1}{\sqrt{ 2 } }\left[ i\alpha  \ket{11} +\alpha e^{ i 3\pi/4 }\ket{01}+\beta \ket{00} +\beta e^{ i\pi/4 }\ket{10}   \right] \\
\]

\hypertarget{c-1}{%
\subsection{c)}\label{c-1}}

We know that the probabilities of measuring a state on one qubit, in an
entangled system, is the sum of the probabilities of possible states. We
also know probability is conserved, so \[
\mathbb{P}(\ket{1} ) = 1-\ket{0} 
\] The probability of measuring \(\ket{1}\) on on the first qubit is \[
\frac{1}{\sqrt{ 2 }}i* \frac{-i}{\sqrt{ 2 }} = .5
\] Thus our probabilities of measurement are \[
\begin{align}
\mathbb{P}(\ket{1} ) =.5 \\
\mathbb{P}(\ket{0} ) =.5 
\end{align}
\]

\hypertarget{d}{%
\subsection{d)}\label{d}}

When the first qubit is measured as a \(\ket{0}\), then our resulting
state becomes \[
\ket{\psi_{out}} \frac{k}{\sqrt{ 2 }}\left[  i(\alpha+\beta) e^{ i\pi/4 }\ket{1}+\beta \ket{0}  \right] 
\] The state itself can be then written as \[
\ket{\psi_{out}} = ae^{ i 3\pi/4 }\ket{1} +b\ket{0} 
\]

\begin{center}\rule{0.5\linewidth}{0.5pt}\end{center}

When the first qubit is measured as \(\ket{1}\), then the second qubit
can only be \[
\ket{\psi_{out}} = be^{ i\pi/4 }\ket{0} +ia\ket{1} 
\] Where \(a,b\) are normalized constants for the states

\hypertarget{f}{%
\subsection{f)}\label{f}}

\begin{figure}
\centering
\includegraphics{../../../Supplemental Files/images/Pasted image 20240922194846.png}
\caption{Quantum Composer Circuit}
\end{figure}

\begin{figure}
\centering
\includegraphics{../../../Supplemental Files/images/Pasted image 20240925210108.png}
\caption{Histogram of Results}
\end{figure}

Unseen in the histogram is the counts, which were

\begin{longtable}[]{@{}ll@{}}
\toprule\noalign{}
State & Counts \\
\midrule\noalign{}
\endhead
\bottomrule\noalign{}
\endlastfoot
\(\ket{11}\) & 524 \\
\(\ket{10}\) & 458 \\
\(\ket{01}\) & 24 \\
\(\ket{00}\) & 18 \\
\end{longtable}

\hypertarget{g}{%
\subsection{g)}\label{g}}

\[
TV = \frac{\left| 512-524 \right| +\left| 512-458 \right| +24+18}{2*1024}  = \frac{27}{512}
\]

\hypertarget{problem-4}{%
\section{Problem 4}\label{problem-4}}

\hypertarget{a-2}{%
\subsection{a)}\label{a-2}}

\begin{figure}
\centering
\includegraphics{../../../Supplemental Files/images/Pasted image 20240922195746.png}
\caption{Quantum Circuit}
\end{figure}

\[
\ket{\psi}  = \frac{{\ket{01} -\ket{10} }}{\sqrt{ 2 }}
\]

\hypertarget{b-2}{%
\subsection{b)}\label{b-2}}

\begin{figure}
\centering
\includegraphics{../../../Supplemental Files/images/Pasted image 20240925205810.png}
\caption{Measurement Outcome Data}
\end{figure}

\begin{longtable}[]{@{}ll@{}}
\toprule\noalign{}
\textbf{Measurement outcome} & \textbf{Frequency} \\
\midrule\noalign{}
\endhead
\bottomrule\noalign{}
\endlastfoot
\textbf{11} & 113 \\
\textbf{10} & 429 \\
\textbf{01} & 465 \\
00 & 17 \\
\end{longtable}

\hypertarget{c-2}{%
\subsection{c)}\label{c-2}}

\[
TV = \frac{\left| 512-429 \right| +\left| 512-465 \right| +17+113}{2*1024}  =\frac{65}{512}
\]

\newpage
\hypertarget{problem-5}{%
\section{Problem 5}\label{problem-5}}

\[
\begin{align*}  
\ket{000} & \mapsto \frac{\sqrt{3}}{2} \ket{000} + \frac{\ket{110} + \ket{101} + \ket{011}}{\sqrt{12}} \\  
\ket{001} & \mapsto \frac{\sqrt{3}}{2} \ket{111} + \frac{\ket{001} + \ket{010} + \ket{100}}{\sqrt{12}} \\  
\end{align*}
\] Given this unitary, we already know that our chance of successfully
cloning \(\left\{ \ket{0},\ket{1} \right\}\) is \(\frac{3}{4}\). To find
the probability of success on \(\left\{\ket{+},\ket{-} \right\}\), we
use the fact that \[
\ket{+}  = \frac{1}{\sqrt{ 2 }}\left[ \ket{0} +\ket{1}  \right] 
\] And \[
\begin{align}
\ket{00+}  = \frac{1}{\sqrt{ 2 }}\left( \ket{001} +\ket{000}  \right)  \\
=\ket{00} \otimes \frac{1}{\sqrt{ 2 }}\left( \ket{1} +\ket{0}  \right) 
\end{align}
\] Then applying this fact to our plus-minus states gives \[
\begin{align}
\ket{00+}  \to  \frac{1}{\sqrt{ 2 }}\left[ \frac{\sqrt{3}}{2} \boxed{\ket{000} } + \frac{\ket{110} + \ket{101} + \ket{011}}{\sqrt{12}} \right] +\frac{1}{\sqrt{ 2 }}\left[ \frac{\sqrt{3}}{2} \boxed{\ket{111} } + \frac{\ket{001} + \ket{010} + \ket{100}}{\sqrt{12}} \right]  \\
\ket{00-}  \to  \frac{1}{\sqrt{ 2 }}\left[ \frac{\sqrt{3}}{2} \boxed{\ket{000} } + \frac{\ket{110} + \ket{101} + \ket{011}}{\sqrt{12}} \right] -\frac{1}{\sqrt{ 2 }}\left[ \frac{\sqrt{3}}{2} \boxed{\ket{111} } + \frac{\ket{001} + \ket{010} + \ket{100}}{\sqrt{12}} \right] 
\end{align}
\]

Our boxed terms when combined, represent successes of the state as below
\[
\begin{align}
\frac{\sqrt{ 3 }}{2\sqrt{ 2 }}\ket{000}+\frac{\sqrt{ 3 }}{2\sqrt{ 2 }}\ket{111}  \\
\frac{\sqrt{ 3 }}{2}\left( \frac{1}{\sqrt{ 2 }}\ket{000} +\frac{1}{\sqrt{ 2 } }\ket{111} \right)  = \frac{\sqrt{ 3 }}{2}\left( \ket{+++}  \right) 
\end{align}
\] This gives us a minimum success as \(\frac{3}{8}\).

\[
\begin{align}
\ket{101}  = \ket{1} \otimes \ket{1} \otimes \ket{0}  = \frac{1}{\sqrt{ 2 }}\left[ \left( \ket{+} -\ket{-}  \right)\otimes\left( \ket{+} -\ket{-}  \right)\otimes \left( \ket{+} +\ket{-}  \right)  \right]  \\
\frac{1}{\sqrt{ 2 }}\left[ \ket{+++} -\ket{+--}-\ket{-++}+\ket{---}    \right] 
\end{align}
\]

\begin{Shaded}
\begin{Highlighting}[]
\NormalTok{qubit = input("What\textbackslash{}\textquotesingle{}s the qubit?")}
\NormalTok{zero,one = sp.symbols(\textquotesingle{}z i\textquotesingle{})}
\NormalTok{plus, minus = sp.symbols(\textquotesingle{}X mu\textquotesingle{})}


\NormalTok{ket = []}
\NormalTok{for bit in qubit:}
\NormalTok{    if bit == \textquotesingle{}1\textquotesingle{}:}
\NormalTok{        ket.append(one)}
\NormalTok{    elif bit == \textquotesingle{}0\textquotesingle{}:}
\NormalTok{        ket.append(zero)}
\NormalTok{ket = sp.Matrix(ket)}
\NormalTok{ket = ket.subs(\{zero:plus+minus,one:plus{-}minus\})}
\NormalTok{sp.pprint(ket)}
\end{Highlighting}
\end{Shaded}

\begin{longtable}[]{@{}
  >{\raggedright\arraybackslash}p{(\columnwidth - 2\tabcolsep) * \real{0.1954}}
  >{\raggedright\arraybackslash}p{(\columnwidth - 2\tabcolsep) * \real{0.8046}}@{}}
\toprule\noalign{}
\begin{minipage}[b]{\linewidth}\raggedright
\(\left\{ \ket{0},\ket{1} \right\}\)
\end{minipage} & \begin{minipage}[b]{\linewidth}\raggedright
\(\left\{ \ket{+},\ket{-} \right\}\)
\end{minipage} \\
\midrule\noalign{}
\endhead
\bottomrule\noalign{}
\endlastfoot
*\(\ket{001}\) &
\(\frac{\sqrt{2}}{4} \bigg(\ket {+++} + -\ket {++-} + \ket {+-+} + -\ket {+--} + \ket {-++} + -\ket {-+-} + \ket {--+} + -\ket {---}\bigg)\) \\
*\(\ket{010}\) &
\(\frac{\sqrt{2}}{4} \bigg(\ket {+++} + \ket {++-} + -\ket {+-+} + -\ket {+--} + \ket {-++} + \ket {-+-} + -\ket {--+} + -\ket {---}\bigg)\) \\
*\(\ket{100}\) &
\(\frac{\sqrt{2}}{4} \bigg(\ket {+++} + \ket {++-} + \ket {+-+} + \ket {+--} + -\ket {-++} + -\ket {-+-} + -\ket {--+} + -\ket {---}\bigg)\) \\
'\(\ket{011}\) &
\(\frac{\sqrt{2}}{4} \bigg(\ket {+++} + -\ket {++-} + -\ket {+-+} + \ket {+--} + \ket {-++} + -\ket {-+-} + -\ket {--+} + \ket {---}\bigg)\) \\
'\(\ket{101}\) &
\(\frac{\sqrt{2}}{4} \bigg(\ket {+++} + -\ket {++-} + \ket {+-+} + -\ket {+--} + -\ket {-++} + \ket {-+-} + -\ket {--+} + \ket {---}\bigg)\) \\
'\(\ket{110}\) &
\(\frac{\sqrt{2}}{4} \bigg(\ket {+++} + \ket {++-} + -\ket {+-+} + -\ket {+--} + -\ket {-++} + -\ket {-+-} + \ket {--+} + \ket {---}\bigg)\) \\
\(\ket{000}\) &
\(\frac{\sqrt{2}}{4} \bigg(\ket {+++} + \ket {++-} + \ket {+-+} + \ket {+--} + \ket {-++} + \ket {-+-} + \ket {--+} + \ket {---}\bigg)\) \\
\(\ket{111}\) &
\(\frac{\sqrt{2}}{4} \bigg(\ket {+++} + -\ket {++-} + -\ket {+-+} + \ket {+--} + -\ket {-++} + \ket {-+-} + \ket {--+} + -\ket {---}\bigg)\) \\
\end{longtable}

Any combination terms don't matter. For the \(\ket{+}\) case, we add up
the number of terms which give us success. When we copy a \(\ket{+}\),
the \(\ket{---}\) state cancels out. 6 \[
\frac{1}{\cancel{ \sqrt{ 2 } }}* \frac{1}{\sqrt{ 12 }}* \frac{6\cancel{ \sqrt{ 2 } }}{4}\ket{+++}  = \frac{3}{2\sqrt{ 12 }}\ket{+++} 
\] Then in the \(\ket{+}\), \[
P(\ket{+++} ) = \frac{3}{4}+\frac{1}{16}=\frac{13}{16}
\] A symmetry argument can be applied for sending the \(\ket{-}\)

This then gives us a final success probability of \[
P_{tot} = \frac{{P_{\left\{ \ket{0} ,\ket{1}  \right\}  }+P_{\left\{ \ket{+} ,\ket{-}  \right\}  }}}{ 2} = \frac{3}{8}+\frac{13}{32} = \frac{25}{32}
\]
\end{document}