\PassOptionsToPackage{unicode}{hyperref}
\PassOptionsToPackage{hyphens}{url}

\documentclass[]{article}


\usepackage{color}
\usepackage{graphicx}

\usepackage{fancyvrb}
\newcommand{\VerbBar}{|}
\newcommand{\VERB}{\Verb[commandchars=\\\{\}]}
\DefineVerbatimEnvironment{Highlighting}{Verbatim}{commandchars=\\\{\}}
% Add ',fontsize=\small' for more characters per line
\usepackage{framed}
\definecolor{shadecolor}{RGB}{248,248,248}
\newenvironment{Shaded}{\begin{snugshade}}{\end{snugshade}}
\newcommand{\AlertTok}[1]{\textcolor[rgb]{0.94,0.16,0.16}{#1}}
\newcommand{\AnnotationTok}[1]{\textcolor[rgb]{0.56,0.35,0.01}{\textbf{\textit{#1}}}}
\newcommand{\AttributeTok}[1]{\textcolor[rgb]{0.13,0.29,0.53}{#1}}
\newcommand{\BaseNTok}[1]{\textcolor[rgb]{0.00,0.00,0.81}{#1}}
\newcommand{\BuiltInTok}[1]{#1}
\newcommand{\CharTok}[1]{\textcolor[rgb]{0.31,0.60,0.02}{#1}}
\newcommand{\CommentTok}[1]{\textcolor[rgb]{0.56,0.35,0.01}{\textit{#1}}}
\newcommand{\CommentVarTok}[1]{\textcolor[rgb]{0.56,0.35,0.01}{\textbf{\textit{#1}}}}
\newcommand{\ConstantTok}[1]{\textcolor[rgb]{0.56,0.35,0.01}{#1}}
\newcommand{\ControlFlowTok}[1]{\textcolor[rgb]{0.13,0.29,0.53}{\textbf{#1}}}
\newcommand{\DataTypeTok}[1]{\textcolor[rgb]{0.13,0.29,0.53}{#1}}
\newcommand{\DecValTok}[1]{\textcolor[rgb]{0.00,0.00,0.81}{#1}}
\newcommand{\DocumentationTok}[1]{\textcolor[rgb]{0.56,0.35,0.01}{\textbf{\textit{#1}}}}
\newcommand{\ErrorTok}[1]{\textcolor[rgb]{0.64,0.00,0.00}{\textbf{#1}}}
\newcommand{\ExtensionTok}[1]{#1}
\newcommand{\FloatTok}[1]{\textcolor[rgb]{0.00,0.00,0.81}{#1}}
\newcommand{\FunctionTok}[1]{\textcolor[rgb]{0.13,0.29,0.53}{\textbf{#1}}}
\newcommand{\ImportTok}[1]{#1}
\newcommand{\InformationTok}[1]{\textcolor[rgb]{0.56,0.35,0.01}{\textbf{\textit{#1}}}}
\newcommand{\KeywordTok}[1]{\textcolor[rgb]{0.13,0.29,0.53}{\textbf{#1}}}
\newcommand{\NormalTok}[1]{#1}
\newcommand{\OperatorTok}[1]{\textcolor[rgb]{0.81,0.36,0.00}{\textbf{#1}}}
\newcommand{\OtherTok}[1]{\textcolor[rgb]{0.56,0.35,0.01}{#1}}
\newcommand{\PreprocessorTok}[1]{\textcolor[rgb]{0.56,0.35,0.01}{\textit{#1}}}
\newcommand{\RegionMarkerTok}[1]{#1}
\newcommand{\SpecialCharTok}[1]{\textcolor[rgb]{0.81,0.36,0.00}{\textbf{#1}}}
\newcommand{\SpecialStringTok}[1]{\textcolor[rgb]{0.31,0.60,0.02}{#1}}
\newcommand{\StringTok}[1]{\textcolor[rgb]{0.31,0.60,0.02}{#1}}
\newcommand{\VariableTok}[1]{\textcolor[rgb]{0.00,0.00,0.00}{#1}}
\newcommand{\VerbatimStringTok}[1]{\textcolor[rgb]{0.31,0.60,0.02}{#1}}
\newcommand{\WarningTok}[1]{\textcolor[rgb]{0.56,0.35,0.01}{\textbf{\textit{#1}}}}
% Add the geometry package and set smaller margins
\usepackage[margin=1in]{geometry}

\usepackage{amsmath,amssymb}
\usepackage{lmodern}
\usepackage{iftex}
\ifPDFTeX
  \usepackage[T1]{fontenc}
  \usepackage[utf8]{inputenc}
  \usepackage{textcomp} % provide euro and other symbols
\else % if luatex or xetex
  \usepackage{unicode-math} % this also loads fontspec
  \defaultfontfeatures{Scale=MatchLowercase}
  \defaultfontfeatures[\rmfamily]{Ligatures=TeX,Scale=1}
\fi
\usepackage{braket}
\usepackage{fancyhdr}
\usepackage{listings}
\usepackage{framed}
\lstnewenvironment{Highlighting}{}{}
\usepackage{xcolor}
% Use upquote if available, for straight quotes in verbatim environments
\IfFileExists{upquote.sty}{\usepackage{upquote}}{}

\IfFileExists{microtype.sty}{
  \usepackage[]{microtype}
  \UseMicrotypeSet[protrusion]{basicmath} % disable protrusion for tt fonts
}{}

\makeatletter
\@ifundefined{KOMAClassName}{% if non-KOMA class
  \IfFileExists{parskip.sty}{
    \usepackage{parskip}
  }{% else
    \setlength{\parindent}{0pt}
    \setlength{\parskip}{6pt plus 2pt minus 1pt}
  }
}{% if KOMA class
  \KOMAoptions{parskip=half}
}
\makeatother

\usepackage{xcolor}
\usepackage{longtable,booktabs,array}
\usepackage{calc} % for calculating minipage widths

% Correct order of tables after \paragraph or \subparagraph
\usepackage{etoolbox}
\makeatletter
\patchcmd\longtable{\par}{\if@noskipsec\mbox{}\fi\par}{}{}
\makeatother

% Allow footnotes in longtable head/foot
\IfFileExists{footnotehyper.sty}{\usepackage{footnotehyper}}{\usepackage{footnote}}
\makesavenoteenv{longtable}

\setlength{\emergencystretch}{3em} % prevent overfull lines
\providecommand{\tightlist}{%
  \setlength{\itemsep}{0pt}\setlength{\parskip}{0pt}}
\setcounter{secnumdepth}{-\maxdimen} % remove section numbering

\usepackage{framed}
\usepackage{listings}

\ifLuaTeX
  \usepackage{selnolig} % disable illegal ligatures
\fi

\IfFileExists{bookmark.sty}{\usepackage{bookmark}}{\usepackage{hyperref}}
\IfFileExists{xurl.sty}{\usepackage{xurl}}{} % add URL line breaks if available
\urlstyle{same}
\hypersetup{
  hidelinks,
  pdfcreator={LaTeX via pandoc}
}

% Custom header and footer
\pagestyle{fancy}
\fancyhead[l]{Azal Amer}
\fancyhead[c]{QIS HW \#2}
\fancyhead[r]{\today}
\fancyfoot[c]{\thepage}
\renewcommand{\headrulewidth}{.2pt}
\setlength{\headheight}{15pt}

\begin{document}
\definecolor{shadecolor}{rgb}{0.9, 0.9, 0.9}

\hypertarget{problem-1}{%
\section{Problem 1}\label{problem-1}}

\hypertarget{a}{%
\subsection{a)}\label{a}}

\[
\begin{pmatrix}
0 & -i \\
i & 0
\end{pmatrix}
\]

\hypertarget{b}{%
\subsection{b)}\label{b}}

Answer is \[
\begin{pmatrix}
0 & -1 & 1 & -1 \\
1 & 0 & -1 & -1 \\
-1 & 1 & 0 & -1 \\
1 & 1 & 1 & 0
\end{pmatrix}
\]

Unitary matricies are orthogonal

\begin{Shaded}
\begin{Highlighting}[]
\NormalTok{from sympy import Matrix}
\NormalTok{import sympy as sp}
\NormalTok{matrix =sp.Rational(1)/sp.sqrt(3)* Matrix([}
\NormalTok{    [0, {-}1, 1, {-}1],}
\NormalTok{    [1, 0, {-}1, {-}1],}
\NormalTok{    [{-}1, 1, 0, {-}1],}
\NormalTok{    [1, 1, 1, 0],}

\NormalTok{])}
\NormalTok{sp.pprint(matrix.inv())}
\NormalTok{sp.pprint(matrix.T)}
\end{Highlighting}
\end{Shaded}

\hypertarget{c}{%
\subsection{c)}\label{c}}

\[
\begin{pmatrix}
\alpha_{1} & \alpha_{2} & \alpha_{3} \\
\beta_{1} & \beta_{2} & \beta_{3} \\
\gamma_{1} & \gamma_{2} & \gamma_{3}
\end{pmatrix}
\] The diagonals and anti-diagonals have the below conditions \[
\begin{align}
D = \left\{ n\bigg| n=0\right\}  \\
A = \left\{ u\bigg|u\neq 0 \right\} 
\end{align}
\] Through the definition of the diagonal and anti-diagonal, we also
know the below must be true. \[
\begin{align}
\alpha_{1},\beta_{2}.\gamma_{3}\in D \\
\alpha_{3} , \beta_{2} , \gamma_{1} \in A \\
\beta_{2}\in A,D
\end{align}
\] This is a contradiction, as \(\beta_{2}\) cannot be equal to zero and
unequal to zero. One of the two conditions cannot be satisfied.

\hypertarget{problem-2}{%
\section{Problem 2}\label{problem-2}}

Below are the defined matricies

\begin{Shaded}
\begin{Highlighting}[]
\NormalTok{import sympy as sp}

\NormalTok{\# Define the imaginary unit}
\NormalTok{i = sp.I}

\NormalTok{\# Define the matrices}
\NormalTok{X = sp.Matrix([[0, 1], [1, 0]])}
\NormalTok{Y = sp.Matrix([[0, {-}i], [i, 0]])}
\NormalTok{Z = sp.Matrix([[1, 0], [0, {-}1]])}

\NormalTok{H = sp.Matrix([[1, 1], [1, {-}1]]) / sp.sqrt(2)}
\NormalTok{R\_pi\_4 = sp.Matrix([[1, {-}1], [1, 1]]) / sp.sqrt(2)}
\NormalTok{P = sp.Matrix([[1, 0], [0, i]])}
\NormalTok{T = sp.Matrix([[1, 0], [0, sp.exp(i*sp.pi/4)]])}
\NormalTok{ketH = sp.Matrix([1,0])}
\end{Highlighting}
\end{Shaded}

\hypertarget{a-1}{%
\subsection{a)}\label{a-1}}

\begin{Shaded}
\begin{Highlighting}[]
\NormalTok{sp.pprint(P*H*Z*H*ketH)}
\end{Highlighting}
\end{Shaded}

Thus this gate transforms the \(\ket{0} \to i\ket{1}\)

\hypertarget{b-1}{%
\subsection{b)}\label{b-1}}

\begin{Shaded}
\begin{Highlighting}[]
\NormalTok{sp.pprint(H*Y*Z*R\_pi\_4*ketH)}
\end{Highlighting}
\end{Shaded}

\[
\ket{0}\to i\ket{0}  
\]

\hypertarget{c-1}{%
\subsection{c)}\label{c-1}}

To measure in the \(\left\{ \ket{+},\ket{-} \right\}\) basis, we need to
measure using the \(X\) observable, so we just add that as a gate

\begin{Shaded}
\begin{Highlighting}[]
\NormalTok{sp.pprint(H*T*X*ketH)}
\end{Highlighting}
\end{Shaded}

\[
\ket{\phi}  = \frac{\sqrt{ 2 }}{2}e^{ i\pi/4 }\ket{0} -\frac{\sqrt{ 2 }}{2}e^{ i\pi/4 }\ket{-} 
\]

\hypertarget{d}{%
\subsection{d)}\label{d}}

To measure in the \(\left\{ \ket{i},\ket{-i} \right\}\) basis, we use
the \(Y\) observable

\begin{Shaded}
\begin{Highlighting}[]
\NormalTok{sp.pprint(T*Z*T*Y*ketH)}
\end{Highlighting}
\end{Shaded}

\[
\ket{\psi}  = 0\ket{i} +\ket{-i}  
\]

\hypertarget{problem-3}{%
\section{Problem 3}\label{problem-3}}

\hypertarget{a-2}{%
\subsection{a)}\label{a-2}}

\begin{Shaded}
\begin{Highlighting}[]
\NormalTok{ketPlus = sp.Rational(1)/sp.sqrt(2) *sp.Matrix(}
\NormalTok{[1,1]}
\NormalTok{)}


\NormalTok{psi = ketH+ketPlus}
\NormalTok{sp.pprint((psi.T*psi).expand())}
\end{Highlighting}
\end{Shaded}

To normalize any vector, we just divide it by the square root of it's
own inner product. \[
\ket{\psi}  = c\left( \ket{0} +\ket{+}  \right) 
\] Where \(c\) is the normalization constant \[
c = \frac{1}{\sqrt{ \left<\psi|\psi\right>  }}
\] \[
c = \frac{1}{\sqrt{ \sqrt{ 2 }+2 }}
\] Therefore the normalized state is \[
\ket{\psi}  = \frac{1}{\sqrt{ \sqrt{ 2 }+2 }}\left( \ket{0} +\ket{+}  \right) =\frac{1}{\sqrt{ \sqrt{ 2 }+2 }}\begin{pmatrix}
1+\frac{1}{\sqrt{ 2 }}\\
\frac{1}{\sqrt{ 2 }}
\end{pmatrix}
\]

\hypertarget{b-2}{%
\subsection{b)}\label{b-2}}

\begin{Shaded}
\begin{Highlighting}[]
\NormalTok{print(sp.latex(H.eigenvects()))}
\end{Highlighting}
\end{Shaded}

The eigenvectors of the Hadamard gate are \[
 \left( -1,\  \left[\begin{matrix}1 - \sqrt{2}\\1\end{matrix}\right]=\vec{v_{1}}\right), \ \left( 1, \left[\begin{matrix}1 + \sqrt{2}\\1\end{matrix}=\vec{v_{2}}\right]\right)
\] It seems the distinguishing component between both vectors is the
\(\ket{0}\) amplitude, so we need to check if the first component of
each vector is equal, when the second component is 1.

\[
\begin{align}
\sqrt{ 2 }\left( 1+\frac{1}{\sqrt{ 2 }} \right) =^{?}1+\sqrt{ 2 } \\
\sqrt{ 2 }+1=1+\sqrt{ 2 } \\
\frac{1}{\sqrt{ \sqrt{ 2 }+2 }}\left( \ket{0} +\ket{+}  \right) =\vec{v_{1}}
\end{align} \\
\]

\hypertarget{c-2}{%
\subsection{c)}\label{c-2}}

The \(P\) gate moves is through a complex rotation, orthogonal to motion
by the \(H\) gate on the Bloch Sphere. We only have two possible states
of motion.

\includegraphics{../../../Supplemental Files/images/Pasted image 20240917144240.png}
We can only make orthogonal rotations, in orthogonal directions. From
this, it means we can only reach pure states on the axis times the
negative pair.

\hypertarget{problem-4}{%
\section{Problem 4}\label{problem-4}}

\hypertarget{a-3}{%
\subsection{a)}\label{a-3}}

It's orthogonal eigenvectors are equally distant from
\(\ket{+},\ket{0}\). The bisecting vector between \(\ket{0},\ket{+}\) is
just \[
\ket{\psi}  = \sin(\theta)\ket{0} +\cos(\theta)\ket{1} 
\]

We need to find the matrix which describes this rotation, for one vector
apply it twice, for the other apply it's inverse. We know that \[
U*U*U*U =\begin{pmatrix}
0 & 1 \\
1 & 0
\end{pmatrix}
\]

Which means that \(U\) applies a rotation of \(\frac{\pi}{2}/4\) each
time. This becomes \(\frac{\pi}{8}\). We know that both our vectors are
real, which means our unitary is represented by \[
U_{\mathbb{R}} = \begin{pmatrix}
\cos \theta & -\sin \theta \\
\sin \theta & \cos \theta
\end{pmatrix}
\] Where \(\theta\) is the rotation we want to apply. \[
U = \begin{pmatrix}
\cos \frac{\pi}{8} & -\sin \frac{\pi}{8} \\
\sin \frac{\pi}{8} & \cos \frac{\pi}{8}
\end{pmatrix}
\]

We want to measure in this observable. For a state \(\ket{\psi}\), one
eigenvector having the high number of counts corresponds to the
\(\ket{0}\) state, and the other corresponds to \(\ket{+}\). A detection
with the eigenvalue corresponding to \(\lambda=1\) (in this case, on the
\(\ket{0}\) detector) has an expected value of \[
\begin{align}
\left| \left<0|U|0\right>  \right| ^{2}= \begin{pmatrix}
1 & 0 
\end{pmatrix}\begin{pmatrix}
\cos \frac{\pi}{8} & -\sin \frac{\pi}{8} \\
\sin \frac{\pi}{8} & \cos \frac{\pi}{8}
\end{pmatrix}\begin{pmatrix}
1 \\
0
\end{pmatrix} \\
\begin{pmatrix}
\cos \frac{\pi}{8}  & -\sin \frac{\pi}{8}
\end{pmatrix}\begin{pmatrix}
1 \\
0
\end{pmatrix}=\cos^{2} \frac{\pi}{8}
\end{align}
\] With \(\ket{+}\), our expected value is \[
\begin{align}
\left| \left<+|U|+\right>  \right| ^{2}= \frac{1}{2} \begin{pmatrix}
1 & 1 
\end{pmatrix}\begin{pmatrix}
\cos \frac{\pi}{8} & -\sin \frac{\pi}{8} \\
\sin \frac{\pi}{8} & \cos \frac{\pi}{8}
\end{pmatrix}\begin{pmatrix}
1 \\
1
\end{pmatrix} \\
\frac{1}{2}\begin{pmatrix}
\cos \frac{\pi}{8}+\sin \frac{\pi}{8}  & \cos \frac{\pi}{8}-\sin \frac{\pi}{8}
\end{pmatrix}\begin{pmatrix}
1 \\
1
\end{pmatrix}= \\
=\left( \frac{1}{2} 2\cos \frac{\pi}{8} \right) ^{2}=\cos^{2} \frac{\pi}{8}
\end{align}
\] The expected value of both of these states measuring in the
observable is \(\frac{\cos^{2}\pi}{8}\). We know that \[
\begin{align}
1 = \text{success}+\text{error} \\
1=\cos^{2} \frac{\pi}{8}+\text{error}\\ \\
1-\cos^{2} \frac{\pi}{8}=\text{error} \\
\text{error} = \sin^{2}{ \frac{\pi}{8} } 
\end{align}
\]

\hypertarget{c-3}{%
\subsection{c)}\label{c-3}}

Measuring in the \(\left\{ \ket{0},\ket{1} \right\}\) basis instead of
our proper basis, gives an error on average, of \[
\frac{{e_{\ket{0}}+e_{\ket{+} }}}{2}=e_{tot}
\] We know that we have a click on detector \(\ket{1}\), then we have to
be diagonal. \[
\ket{+}  = \frac{1}{\sqrt{ 2 } }\ket{0} +\frac{1}{\sqrt{ 2 }}\ket{1} 
\] The diagonal state 50\% of the time collapses to \(\ket{0}\). So \[
e_{\ket{+} } = .5
\] \(\ket{0}\) will always collapse to zero, so \[
e_{\ket{0} } = 1
\] Therefore, \[
e_{tot} = \frac{{1+.5}}{2} = .75
\]
\end{document}