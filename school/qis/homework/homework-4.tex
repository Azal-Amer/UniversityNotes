\PassOptionsToPackage{unicode}{hyperref}
\PassOptionsToPackage{hyphens}{url}

\documentclass[]{article}


\usepackage{color}
\usepackage{graphicx}

\usepackage{fancyvrb}
\newcommand{\VerbBar}{|}
\newcommand{\VERB}{\Verb[commandchars=\\\{\}]}
\DefineVerbatimEnvironment{Highlighting}{Verbatim}{commandchars=\\\{\}}
% Add ',fontsize=\small' for more characters per line
\usepackage{framed}
\definecolor{shadecolor}{RGB}{248,248,248}
\newenvironment{Shaded}{\begin{snugshade}}{\end{snugshade}}
\newcommand{\AlertTok}[1]{\textcolor[rgb]{0.94,0.16,0.16}{#1}}
\newcommand{\AnnotationTok}[1]{\textcolor[rgb]{0.56,0.35,0.01}{\textbf{\textit{#1}}}}
\newcommand{\AttributeTok}[1]{\textcolor[rgb]{0.13,0.29,0.53}{#1}}
\newcommand{\BaseNTok}[1]{\textcolor[rgb]{0.00,0.00,0.81}{#1}}
\newcommand{\BuiltInTok}[1]{#1}
\newcommand{\CharTok}[1]{\textcolor[rgb]{0.31,0.60,0.02}{#1}}
\newcommand{\CommentTok}[1]{\textcolor[rgb]{0.56,0.35,0.01}{\textit{#1}}}
\newcommand{\CommentVarTok}[1]{\textcolor[rgb]{0.56,0.35,0.01}{\textbf{\textit{#1}}}}
\newcommand{\ConstantTok}[1]{\textcolor[rgb]{0.56,0.35,0.01}{#1}}
\newcommand{\ControlFlowTok}[1]{\textcolor[rgb]{0.13,0.29,0.53}{\textbf{#1}}}
\newcommand{\DataTypeTok}[1]{\textcolor[rgb]{0.13,0.29,0.53}{#1}}
\newcommand{\DecValTok}[1]{\textcolor[rgb]{0.00,0.00,0.81}{#1}}
\newcommand{\DocumentationTok}[1]{\textcolor[rgb]{0.56,0.35,0.01}{\textbf{\textit{#1}}}}
\newcommand{\ErrorTok}[1]{\textcolor[rgb]{0.64,0.00,0.00}{\textbf{#1}}}
\newcommand{\ExtensionTok}[1]{#1}
\newcommand{\FloatTok}[1]{\textcolor[rgb]{0.00,0.00,0.81}{#1}}
\newcommand{\FunctionTok}[1]{\textcolor[rgb]{0.13,0.29,0.53}{\textbf{#1}}}
\newcommand{\ImportTok}[1]{#1}
\newcommand{\InformationTok}[1]{\textcolor[rgb]{0.56,0.35,0.01}{\textbf{\textit{#1}}}}
\newcommand{\KeywordTok}[1]{\textcolor[rgb]{0.13,0.29,0.53}{\textbf{#1}}}
\newcommand{\NormalTok}[1]{#1}
\newcommand{\OperatorTok}[1]{\textcolor[rgb]{0.81,0.36,0.00}{\textbf{#1}}}
\newcommand{\OtherTok}[1]{\textcolor[rgb]{0.56,0.35,0.01}{#1}}
\newcommand{\PreprocessorTok}[1]{\textcolor[rgb]{0.56,0.35,0.01}{\textit{#1}}}
\newcommand{\RegionMarkerTok}[1]{#1}
\newcommand{\SpecialCharTok}[1]{\textcolor[rgb]{0.81,0.36,0.00}{\textbf{#1}}}
\newcommand{\SpecialStringTok}[1]{\textcolor[rgb]{0.31,0.60,0.02}{#1}}
\newcommand{\StringTok}[1]{\textcolor[rgb]{0.31,0.60,0.02}{#1}}
\newcommand{\VariableTok}[1]{\textcolor[rgb]{0.00,0.00,0.00}{#1}}
\newcommand{\VerbatimStringTok}[1]{\textcolor[rgb]{0.31,0.60,0.02}{#1}}
\newcommand{\WarningTok}[1]{\textcolor[rgb]{0.56,0.35,0.01}{\textbf{\textit{#1}}}}
% Add the geometry package and set smaller margins
\usepackage[margin=1in]{geometry}

\usepackage{amsmath,amssymb}
\usepackage{lmodern}
\usepackage{iftex}
\ifPDFTeX
  \usepackage[T1]{fontenc}
  \usepackage[utf8]{inputenc}
  \usepackage{textcomp} % provide euro and other symbols
\else % if luatex or xetex
  \usepackage{unicode-math} % this also loads fontspec
  \defaultfontfeatures{Scale=MatchLowercase}
  \defaultfontfeatures[\rmfamily]{Ligatures=TeX,Scale=1}
\fi
\usepackage{braket}
\usepackage{fancyhdr}
\usepackage{listings}
\usepackage{framed}
\lstnewenvironment{Highlighting}{}{}
\usepackage{xcolor}
% Use upquote if available, for straight quotes in verbatim environments
\IfFileExists{upquote.sty}{\usepackage{upquote}}{}

\IfFileExists{microtype.sty}{
  \usepackage[]{microtype}
  \UseMicrotypeSet[protrusion]{basicmath} % disable protrusion for tt fonts
}{}

\makeatletter
\@ifundefined{KOMAClassName}{% if non-KOMA class
  \IfFileExists{parskip.sty}{
    \usepackage{parskip}
  }{% else
    \setlength{\parindent}{0pt}
    \setlength{\parskip}{6pt plus 2pt minus 1pt}
  }
}{% if KOMA class
  \KOMAoptions{parskip=half}
}
\makeatother

\usepackage{xcolor}
\usepackage{longtable,booktabs,array}
\usepackage{calc} % for calculating minipage widths

% Correct order of tables after \paragraph or \subparagraph
\usepackage{etoolbox}
\makeatletter
\patchcmd\longtable{\par}{\if@noskipsec\mbox{}\fi\par}{}{}
\makeatother

% Allow footnotes in longtable head/foot
\IfFileExists{footnotehyper.sty}{\usepackage{footnotehyper}}{\usepackage{footnote}}
\makesavenoteenv{longtable}

\setlength{\emergencystretch}{3em} % prevent overfull lines
\providecommand{\tightlist}{%
  \setlength{\itemsep}{0pt}\setlength{\parskip}{0pt}}
\setcounter{secnumdepth}{-\maxdimen} % remove section numbering

\usepackage{framed}
\usepackage{listings}

\ifLuaTeX
  \usepackage{selnolig} % disable illegal ligatures
\fi

\IfFileExists{bookmark.sty}{\usepackage{bookmark}}{\usepackage{hyperref}}
\IfFileExists{xurl.sty}{\usepackage{xurl}}{} % add URL line breaks if available
\urlstyle{same}
\hypersetup{
  hidelinks,
  pdfcreator={LaTeX via pandoc}
}

% Custom header and footer
\pagestyle{fancy}
\fancyhead[l]{Azal Amer}
\fancyhead[c]{QIS Homework  \#4}
\fancyhead[r]{\today}
\fancyfoot[c]{\thepage}
\renewcommand{\headrulewidth}{.2pt}
\setlength{\headheight}{15pt}

\begin{document}
\definecolor{shadecolor}{rgb}{0.9, 0.9, 0.9}

\hypertarget{problem-1}{%
\section{Problem 1}\label{problem-1}}

\hypertarget{a}{%
\subsection{a)}\label{a}}

To find the probability of measuring any qubit in a certain basis, note
that \[
\begin{align}
\ket{0}  = \frac{\sqrt{ 2 }}{2}(\ket{+} +\ket{-}) \\
\ket{1}  = \frac{\sqrt{ 2 }}{2}(\ket{+} -\ket{-}  )
\end{align}
\] We can then use this fact inside of our wave-function \(\ket{\psi}\)
\[
\begin{align}
\ket{\psi} = \frac{{\ket{00} +\sqrt{ 2 }\ket{10} +\sqrt{ 2 }\ket{01} - \ket{11} }}{\sqrt{ 6 }} \\
 = \frac{{\ket{0}\otimes \ket{0}  +\sqrt{ 2 }(\ket{1}\otimes \ket{0})  +\sqrt{ 2 }(\ket{0}\otimes \ket{1})  - \ket{1}\otimes \ket{1}  }}{\sqrt{ 6 }} 
\end{align}
\]

Visually, this produces the following map

\begin{longtable}[]{@{}ll@{}}
\toprule\noalign{}
Digital & Plus-Minus \\
\midrule\noalign{}
\endhead
\bottomrule\noalign{}
\endlastfoot
\(\ket{00}\) & \(\ket{++}+\ket{+ -} +\ket{- +}+\ket{--}\) \\
\(\ket{10}\) & \(\ket{++}+\ket{+ -}-\ket{- +}-\ket{--}\) \\
\(\ket{01}\) & \(\ket{++}-\ket{+ -}+\ket{- +}-\ket{--}\) \\
\(\ket{11}\) & \(\ket{++}-\ket{+ -} -\ket{- +}+\ket{--}\) \\
\end{longtable}

\[
\begin{align}
\ket{\psi}  = \frac{\sqrt{ 2 }  \ket{++}+\ket{+-}+\ket{-+}+ \sqrt{ 2 }  \ket{--}   }{\sqrt{ 6 }}
\end{align}
\] The probability of measuring the first qubit in the plus state, is\\
\[
\left( \frac{\sqrt{ 2 }}{\sqrt{ 6 }} \right)^{2}+\left( \frac{1}{\sqrt{ 6 }} \right)^{2} = \frac{1}{2}
\] \#\# b) \[
\ket{\psi'}  = \frac{\sqrt{ 2 }\ket{+} +\ket{-} }{\sqrt{ 3 }} 
\]

\hypertarget{c}{%
\subsection{c)}\label{c}}

\[
P = \left( \frac{1}{\sqrt{ 6 }}  \right)^{2}= \frac{1}{6}
\]

\hypertarget{problem-2}{%
\section{Problem 2}\label{problem-2}}

\[
\ket{EPR}  = \frac{{\ket{00} +\ket{11} }}{\sqrt{ 2 }}
\] We know that this is equivalent in the
\(\left\{ \ket{+},\ket{-} \right\}\) \[
\ket{EPR}  = \frac{{\ket{++} +\ket{--} }}{\sqrt{ 2 }}
\] Alice can encode her single bit of information inside the measurement
of what basis she uses. Whatever state she gets when she measures, will
be sent to Bob. Bob then needs to clone his qubit several times to
perform several measurements to confirm the basis Alice used. Bob can
just see which basis gives him the consistent answer, and the
probability of his correct success is based on the quantity of cloning.

\hypertarget{problem-3}{%
\section{Problem 3}\label{problem-3}}

\hypertarget{a-1}{%
\subsection{a)}\label{a-1}}

Alice's qubit becomes

\begin{longtable}[]{@{}
  >{\raggedright\arraybackslash}p{(\columnwidth - 12\tabcolsep) * \real{0.2286}}
  >{\raggedright\arraybackslash}p{(\columnwidth - 12\tabcolsep) * \real{0.1286}}
  >{\raggedright\arraybackslash}p{(\columnwidth - 12\tabcolsep) * \real{0.1286}}
  >{\raggedright\arraybackslash}p{(\columnwidth - 12\tabcolsep) * \real{0.1286}}
  >{\raggedright\arraybackslash}p{(\columnwidth - 12\tabcolsep) * \real{0.1286}}
  >{\raggedright\arraybackslash}p{(\columnwidth - 12\tabcolsep) * \real{0.1286}}
  >{\raggedright\arraybackslash}p{(\columnwidth - 12\tabcolsep) * \real{0.1286}}@{}}
\toprule\noalign{}
\begin{minipage}[b]{\linewidth}\raggedright
b
\end{minipage} & \begin{minipage}[b]{\linewidth}\raggedright
1
\end{minipage} & \begin{minipage}[b]{\linewidth}\raggedright
0
\end{minipage} & \begin{minipage}[b]{\linewidth}\raggedright
1
\end{minipage} & \begin{minipage}[b]{\linewidth}\raggedright
0
\end{minipage} & \begin{minipage}[b]{\linewidth}\raggedright
1
\end{minipage} & \begin{minipage}[b]{\linewidth}\raggedright
1
\end{minipage} \\
\midrule\noalign{}
\endhead
\bottomrule\noalign{}
\endlastfoot
a & 0 & 1 & 1 & 0 & 0 & 1 \\
\(\ket{\psi_{i}}\) & \(\ket{+}\) & \(\ket{1}\) & \(\ket{-}\) &
\(\ket{0}\) & \(\ket{+}\) & \(\ket{-}\) \\
\end{longtable}

\hypertarget{b}{%
\subsection{b)}\label{b}}

\begin{longtable}[]{@{}
  >{\raggedright\arraybackslash}p{(\columnwidth - 12\tabcolsep) * \real{0.1566}}
  >{\raggedright\arraybackslash}p{(\columnwidth - 12\tabcolsep) * \real{0.1084}}
  >{\raggedright\arraybackslash}p{(\columnwidth - 12\tabcolsep) * \real{0.1084}}
  >{\raggedright\arraybackslash}p{(\columnwidth - 12\tabcolsep) * \real{0.2048}}
  >{\raggedright\arraybackslash}p{(\columnwidth - 12\tabcolsep) * \real{0.2048}}
  >{\raggedright\arraybackslash}p{(\columnwidth - 12\tabcolsep) * \real{0.1084}}
  >{\raggedright\arraybackslash}p{(\columnwidth - 12\tabcolsep) * \real{0.1084}}@{}}
\toprule\noalign{}
\begin{minipage}[b]{\linewidth}\raggedright
\(b'\)
\end{minipage} & \begin{minipage}[b]{\linewidth}\raggedright
1
\end{minipage} & \begin{minipage}[b]{\linewidth}\raggedright
0
\end{minipage} & \begin{minipage}[b]{\linewidth}\raggedright
0
\end{minipage} & \begin{minipage}[b]{\linewidth}\raggedright
1
\end{minipage} & \begin{minipage}[b]{\linewidth}\raggedright
1
\end{minipage} & \begin{minipage}[b]{\linewidth}\raggedright
1
\end{minipage} \\
\midrule\noalign{}
\endhead
\bottomrule\noalign{}
\endlastfoot
\(\ket{\psi'}\) & \(\ket{+}\) & \(\ket{1}\) & \(\ket{0},\ket{1}\) &
\(\ket{+},\ket{-}\) & \(\ket{+}\) & \(\ket{-}\) \\
\(a'\) & 0 & 1 & 1 & 1 & 0 & 1 \\
\end{longtable}

\textbf{Note that for two qubits which mismatched, i picked 11}

\hypertarget{c-1}{%
\subsection{c)}\label{c-1}}

\(b_{\perp} = 1011\)

\(a_{\perp} = 0101\)

\begin{longtable}[]{@{}lllllll@{}}
\toprule\noalign{}
\(b_{\perp}\) & 1 & 0 & & & 1 & 1 \\
\midrule\noalign{}
\endhead
\bottomrule\noalign{}
\endlastfoot
\(a_{\perp}\) & 0 & 1 & & & 0 & 1 \\
\end{longtable}

\hypertarget{d}{%
\subsection{d)}\label{d}}

\begin{longtable}[]{@{}
  >{\raggedright\arraybackslash}p{(\columnwidth - 12\tabcolsep) * \real{0.0852}}
  >{\raggedright\arraybackslash}p{(\columnwidth - 12\tabcolsep) * \real{0.1525}}
  >{\raggedright\arraybackslash}p{(\columnwidth - 12\tabcolsep) * \real{0.1525}}
  >{\raggedright\arraybackslash}p{(\columnwidth - 12\tabcolsep) * \real{0.1525}}
  >{\raggedright\arraybackslash}p{(\columnwidth - 12\tabcolsep) * \real{0.1525}}
  >{\raggedright\arraybackslash}p{(\columnwidth - 12\tabcolsep) * \real{0.1525}}
  >{\raggedright\arraybackslash}p{(\columnwidth - 12\tabcolsep) * \real{0.1525}}@{}}
\toprule\noalign{}
\begin{minipage}[b]{\linewidth}\raggedright
\(\ket{\psi_{pair}}\)
\end{minipage} & \begin{minipage}[b]{\linewidth}\raggedright
\(\left\{ \ket{1},\ket{+} \right\}\)
\end{minipage} & \begin{minipage}[b]{\linewidth}\raggedright
\(\left\{ \ket{1},\ket{+} \right\}\)
\end{minipage} & \begin{minipage}[b]{\linewidth}\raggedright
\(\left\{ \ket{0},\ket{-} \right\}\)
\end{minipage} & \begin{minipage}[b]{\linewidth}\raggedright
\(\left\{ \ket{0},\ket{+} \right\}\)
\end{minipage} & \begin{minipage}[b]{\linewidth}\raggedright
\(\left\{ \ket{0},\ket{+} \right\}\)
\end{minipage} & \begin{minipage}[b]{\linewidth}\raggedright
\(\left\{ \ket{1},\ket{-} \right\}\)
\end{minipage} \\
\midrule\noalign{}
\endhead
\bottomrule\noalign{}
\endlastfoot
\end{longtable}

\hypertarget{e}{%
\subsection{e)}\label{e}}

.

\begin{longtable}[]{@{}lllllll@{}}
\toprule\noalign{}
& . & . & . & . & 1 & . \\
\midrule\noalign{}
\endhead
\bottomrule\noalign{}
\endlastfoot
Reason & & & & & & \\
\end{longtable}

\hypertarget{f}{%
\subsection{f)}\label{f}}

Secret key is then 1 from this protocol From the BB84 implementation in
c, it is 0101
\end{document}