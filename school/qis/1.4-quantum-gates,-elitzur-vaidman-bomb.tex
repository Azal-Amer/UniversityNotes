\PassOptionsToPackage{unicode}{hyperref}
\PassOptionsToPackage{hyphens}{url}

\documentclass[]{article}

% Add the geometry package and set smaller margins
\usepackage[margin=1in]{geometry}

\usepackage{amsmath,amssymb}
\usepackage{lmodern}
\usepackage{iftex}
\ifPDFTeX
  \usepackage[T1]{fontenc}
  \usepackage[utf8]{inputenc}
  \usepackage{textcomp} % provide euro and other symbols
\else % if luatex or xetex
  \usepackage{unicode-math} % this also loads fontspec
  \defaultfontfeatures{Scale=MatchLowercase}
  \defaultfontfeatures[\rmfamily]{Ligatures=TeX,Scale=1}
\fi

\usepackage{fancyhdr}

% Use upquote if available, for straight quotes in verbatim environments
\IfFileExists{upquote.sty}{\usepackage{upquote}}{}

\IfFileExists{microtype.sty}{
  \usepackage[]{microtype}
  \UseMicrotypeSet[protrusion]{basicmath} % disable protrusion for tt fonts
}{}

\makeatletter
\@ifundefined{KOMAClassName}{% if non-KOMA class
  \IfFileExists{parskip.sty}{
    \usepackage{parskip}
  }{% else
    \setlength{\parindent}{0pt}
    \setlength{\parskip}{6pt plus 2pt minus 1pt}
  }
}{% if KOMA class
  \KOMAoptions{parskip=half}
}
\makeatother

\usepackage{xcolor}
\usepackage{longtable,booktabs,array}
\usepackage{calc} % for calculating minipage widths

% Correct order of tables after \paragraph or \subparagraph
\usepackage{etoolbox}
\makeatletter
\patchcmd\longtable{\par}{\if@noskipsec\mbox{}\fi\par}{}{}
\makeatother

% Allow footnotes in longtable head/foot
\IfFileExists{footnotehyper.sty}{\usepackage{footnotehyper}}{\usepackage{footnote}}
\makesavenoteenv{longtable}

\setlength{\emergencystretch}{3em} % prevent overfull lines
\providecommand{\tightlist}{%
  \setlength{\itemsep}{0pt}\setlength{\parskip}{0pt}}
\setcounter{secnumdepth}{-\maxdimen} % remove section numbering

\ifLuaTeX
  \usepackage{selnolig} % disable illegal ligatures
\fi

\IfFileExists{bookmark.sty}{\usepackage{bookmark}}{\usepackage{hyperref}}
\IfFileExists{xurl.sty}{\usepackage{xurl}}{} % add URL line breaks if available
\urlstyle{same}
\hypersetup{
  hidelinks,
  pdfcreator={LaTeX via pandoc}
}

% Custom header and footer
\pagestyle{fancy}
\fancyhead[l]{Azal Amer}
\fancyhead[c]{QIS HW \#}
\fancyhead[r]{\today}
\fancyfoot[c]{\thepage}
\renewcommand{\headrulewidth}{.2pt}
\setlength{\headheight}{15pt}

\begin{document}
\hypertarget{overview}{%
\section{Overview}\label{overview}}

\hypertarget{notes}{%
\section{Notes}\label{notes}}

\begin{itemize}
\tightlist
\item
  We don't want to write giant unwieldily matrices, so keeping things
  factored to the tensor product is a good idea
\item
  Quantum Circuits are just a visual notation for expressing a large
  unitary transformation
\end{itemize}

We're just talking about how to go about writing quantum states to a
diagram

\begin{longtable}[]{@{}
  >{\raggedright\arraybackslash}p{(\columnwidth - 0\tabcolsep) * \real{1.0000}}@{}}
\toprule\noalign{}
\begin{minipage}[b]{\linewidth}\raggedright
{[}{[}Supplemental
Files/images/AgACAgEAAxkBAAIBaWbgmVld9QQNQXAZD3krvjMI8VBeAAJGrTEbFpgIR6E9E1x5pS-BAQADAgADeQADNgQ.jpg{]}{]}
\end{minipage} \\
\midrule\noalign{}
\endhead
\bottomrule\noalign{}
\endlastfoot
Photo of how a list of operations maps to a diagram \\
\end{longtable}

\begin{quote}
{[}!question{]}+ How do we get from Dirac notation, all the way up to
wave-particle duality? So far this just feels like a funky particle
\end{quote}

{[}{[}../Quantum 1/1.4 Matrix Representations and
Hamiltonians\#\^{}4385ab{]}{]} There's also the CC-Not, the Toffoli gate
(Page 31 in the textbook). This is an 8x8 matrix, which behaves just
like an AND gate using the two input bits. \[
\ket{x,y,z} \to \ket{x,y,z\oplus xy} 
\] Note that this is different than \(\ket{x,y,xy}\), because unitaries
MUST be reversible. That means you can't just destroy information
(unless you measure).

\begin{quote}
{[}!exercise{]} Assume you have the state
\(\alpha \ket{00}+\beta \ket{01}+\gamma \ket{10}\delta+ \ket{11}\) What
happens if you measure both states at the same time?

\begin{longtable}[]{@{}ll@{}}
\toprule\noalign{}
\(\|\alpha\|^{2}\) & .25 \\
\midrule\noalign{}
\endhead
\bottomrule\noalign{}
\endlastfoot
\(\|\beta\|^{2}\) & .25 \\
\(\|\gamma\|^{2}\) & .25 \\
\(\|\delta\|^{2}\) & .25 \\
\end{longtable}

But now what happens when you measure them one after the other? We're
going to measure the first qubit.

\begin{longtable}[]{@{}ll@{}}
\toprule\noalign{}
\(P(0)\) & \(\|\alpha\|^{2}+\|\beta\|^{2}\) \\
\midrule\noalign{}
\endhead
\bottomrule\noalign{}
\endlastfoot
P(1) & \(\|\gamma\|^{2}+\|\delta\|^{2}\) \\
\end{longtable}

What the new state of the second qubit is, then affects what the output
of the first qubit is.
\end{quote}

\begin{quote}
{[}!question{]}+ What happens if I measure the first qubit in a
different basis that isn't the digital basis? I think you just rewrite
the basis of the first qubit into the new basis, write the possible
states after measurement, then use that to determine your results. That
or you just apply the measurement observables to the qubit you're going
to measure, then work from there.
\end{quote}

I think entanglement finally made sense in my head

\begin{quote}
{[}!definition{]} Watched Pot Effect

\begin{longtable}[]{@{}
  >{\raggedright\arraybackslash}p{(\columnwidth - 0\tabcolsep) * \real{1.0000}}@{}}
\toprule\noalign{}
\begin{minipage}[b]{\linewidth}\raggedright
{[}{[}Supplemental
Files/images/AgACAgEAAxkBAAIBa2bgoVAEQMG-QVrnmKqqHE2ebDeiAAJNrTEbFpgIR5OyJzNsA1m3AQADAgADeQADNgQ.jpg{]}{]}
\end{minipage} \\
\midrule\noalign{}
\endhead
\bottomrule\noalign{}
\endlastfoot
Image of \(\ket{\alpha}\) drifting without observation \\
\end{longtable}

Let's say we have a vector \(\ket{\alpha}\), and we have our qubit
naturally drifting from the zero to the one state. If we're an
experimentalist, we want to know this is happening, so we keep watching
the qubit, waiting for it to drift.

Let's say the qubit rotates by \(\epsilon\) radians. Nature applies the
below unitary to the qubit continuously \[
\begin{pmatrix}
\cos\epsilon(t)  & -\sin\epsilon(t) \\
\sin\epsilon(t) & \cos\epsilon(t)
\end{pmatrix}\ket{\alpha} 
\] The experimenter will never see the qubit rotate! Without the
measurements, the qubit would be rotating. That being said, the
experimenter measures discretely, which means \[
P\left[ 1 \right]  = \sin^{2}{(\epsilon)} \approx \epsilon^{2}
\] Squaring from the already low probability, means the probability of
the qubit snapping to one is astonishingly small. If we measure
consistently over time and \(\epsilon\) is linear, we make
\(\frac{\pi}{2\epsilon}\) measurements. That means over one rotation,
the probability of snapping to 1 becomes \[
P_{tot}\left[ 1 \right]  = \frac{\pi}{2\epsilon}*P[1] = \frac{\pi}{2}\epsilon
\]
\end{quote}

\begin{quote}
{[}!definition{]} Zeno's Effect Given a qubit now locked at \(\ket{0}\),
and I want to somehow convert it to a \(\ket{1}\) qubit with no gates.
To do that, I can apply measuring in infinite basis with \(d\epsilon\)
angle. That would slowly rotate the darn thing

\begin{quote}
{[}!quote{]} ``Perhaps an everyday-life analog would be asking a
stranger to have coffee with you, then to go dancing, etc.---there's a
higher probability of success than if you just immediately ask them to
marry you!'' - Scott Aaronson
\end{quote}
\end{quote}

\hypertarget{quantum-bomb}{%
\subsection{Quantum Bomb}\label{quantum-bomb}}
\end{document}