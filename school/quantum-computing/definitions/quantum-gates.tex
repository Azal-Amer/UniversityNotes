% Options for packages loaded elsewhere
\PassOptionsToPackage{unicode}{hyperref}
\PassOptionsToPackage{hyphens}{url}
%
\documentclass[
]{article}
\usepackage{amsmath,amssymb}
\usepackage{iftex}
\ifPDFTeX
  \usepackage[T1]{fontenc}
  \usepackage[utf8]{inputenc}
  \usepackage{textcomp} % provide euro and other symbols
\else % if luatex or xetex
  \usepackage{unicode-math} % this also loads fontspec
  \defaultfontfeatures{Scale=MatchLowercase}
  \defaultfontfeatures[\rmfamily]{Ligatures=TeX,Scale=1}
\fi
\usepackage{lmodern}
\ifPDFTeX\else
  % xetex/luatex font selection
\fi
% Use upquote if available, for straight quotes in verbatim environments
\IfFileExists{upquote.sty}{\usepackage{upquote}}{}
\IfFileExists{microtype.sty}{% use microtype if available
  \usepackage[]{microtype}
  \UseMicrotypeSet[protrusion]{basicmath} % disable protrusion for tt fonts
}{}
\makeatletter
\@ifundefined{KOMAClassName}{% if non-KOMA class
  \IfFileExists{parskip.sty}{%
    \usepackage{parskip}
  }{% else
    \setlength{\parindent}{0pt}
    \setlength{\parskip}{6pt plus 2pt minus 1pt}}
}{% if KOMA class
  \KOMAoptions{parskip=half}}
\makeatother
\usepackage{xcolor}
\usepackage{longtable,booktabs,array}
\usepackage{calc} % for calculating minipage widths
% Correct order of tables after \paragraph or \subparagraph
\usepackage{etoolbox}
\makeatletter
\patchcmd\longtable{\par}{\if@noskipsec\mbox{}\fi\par}{}{}
\makeatother
% Allow footnotes in longtable head/foot
\IfFileExists{footnotehyper.sty}{\usepackage{footnotehyper}}{\usepackage{footnote}}
\makesavenoteenv{longtable}
\setlength{\emergencystretch}{3em} % prevent overfull lines
\providecommand{\tightlist}{%
  \setlength{\itemsep}{0pt}\setlength{\parskip}{0pt}}
\setcounter{secnumdepth}{-\maxdimen} % remove section numbering
\ifLuaTeX
  \usepackage{selnolig}  % disable illegal ligatures
\fi
\IfFileExists{bookmark.sty}{\usepackage{bookmark}}{\usepackage{hyperref}}
\IfFileExists{xurl.sty}{\usepackage{xurl}}{} % add URL line breaks if available
\urlstyle{same}
\hypersetup{
  hidelinks,
  pdfcreator={LaTeX via pandoc}}

\author{}
\date{}

\begin{document}

{
\setcounter{tocdepth}{3}
\tableofcontents
}
\textbf{Meet the Gates Family!}

\[
\begin{align*}
I=\begin{bmatrix}
1 & 0  \\ 0 & 1
\end{bmatrix}
 &\quad
  X = \begin{bmatrix}
0 & 1 \\ 1  &  0
\end{bmatrix} 
 &\quad 
S=\begin{bmatrix}
1 & 0  \\ 0 & i
\end{bmatrix} \quad
\\
Y=\begin{bmatrix}
0 & -i \\ i & 0
\end{bmatrix} 
 &\quad
T=\begin{bmatrix}
1 & 0 \\ 0 & e^{\frac{i\pi}{4}}
\end{bmatrix} 
 &\quad 
H =\frac{1}{\sqrt{2}}\begin{bmatrix}
1 & 1  \\ 1 & -1
\end{bmatrix} 
\\
P_0=\begin{bmatrix}
1 & 0 \\ 0  & 0
\end{bmatrix} 
 &\quad
Z=\begin{bmatrix}
1 & 0 \\ 0 & -1
\end{bmatrix} &\quad
S^{\dagger}=\begin{bmatrix}
1 & 0 \\ 0 & -i
\end{bmatrix}
\end{align*}\]

\(X,Y,Z,\& H\) are all different roots of the Identity matrix.

\begin{longtable}[]{@{}llll@{}}
\toprule\noalign{}
Operator & Eigenvalue & Light Polarization & Corresponding Basis \\
\midrule\noalign{}
\endhead
\bottomrule\noalign{}
\endlastfoot
X & +1 & D (Diagonal) & D/A \\
X & -1 & A (Anti-Diagonal) & D/A \\
Y & +1 & R (Right-Circular) & R/L \\
Y & -1 & L (Left-Circular) & R/L \\
Z & +1 & H (Horizontal) & H/V \\
Z & -1 & V (Vertical) & H/V \\
\end{longtable}

\hypertarget{measuring-observable}{%
\section{Measuring Observable}\label{measuring-observable}}

Here's a useful table on what gates to apply to get to the X,Y, and Z
observables from the digital {[}{[}Pasted image 20240319173746.png{]}{]}

\begin{longtable}[]{@{}lll@{}}
\toprule\noalign{}
Observable & Gates & Basis \\
\midrule\noalign{}
\endhead
\bottomrule\noalign{}
\endlastfoot
X & H & D/A \\
Y & \(S H\) & R/L \\
Z & none & H/V \\
\end{longtable}

\end{document}
