\PassOptionsToPackage{unicode}{hyperref}

\PassOptionsToPackage{hyphens}{url}

\documentclass[]{article}

\usepackage{amsmath,amssymb} % Move amsmath and amssymb here

\usepackage{color}
\usepackage{graphicx}

\usepackage{fancyvrb}
\newcommand{\VerbBar}{|}
\newcommand{\VERB}{\Verb[commandchars=\\\{\}]}
\DefineVerbatimEnvironment{Highlighting}{Verbatim}{commandchars=\\\{\}}
% Add ',fontsize=\small' for more characters per line
\usepackage{framed}
\definecolor{shadecolor}{RGB}{248,248,248}
\newenvironment{Shaded}{\begin{snugshade}}{\end{snugshade}}
\newcommand{\AlertTok}[1]{\textcolor[rgb]{0.94,0.16,0.16}{#1}}
\newcommand{\AnnotationTok}[1]{\textcolor[rgb]{0.56,0.35,0.01}{\textbf{\textit{#1}}}}
\newcommand{\AttributeTok}[1]{\textcolor[rgb]{0.13,0.29,0.53}{#1}}
\newcommand{\BaseNTok}[1]{\textcolor[rgb]{0.00,0.00,0.81}{#1}}
\newcommand{\BuiltInTok}[1]{#1}
\newcommand{\CharTok}[1]{\textcolor[rgb]{0.31,0.60,0.02}{#1}}
\newcommand{\CommentTok}[1]{\textcolor[rgb]{0.56,0.35,0.01}{\textit{#1}}}
\newcommand{\CommentVarTok}[1]{\textcolor[rgb]{0.56,0.35,0.01}{\textbf{\textit{#1}}}}
\newcommand{\ConstantTok}[1]{\textcolor[rgb]{0.56,0.35,0.01}{#1}}
\newcommand{\ControlFlowTok}[1]{\textcolor[rgb]{0.13,0.29,0.53}{\textbf{#1}}}
\newcommand{\DataTypeTok}[1]{\textcolor[rgb]{0.13,0.29,0.53}{#1}}
\newcommand{\DecValTok}[1]{\textcolor[rgb]{0.00,0.00,0.81}{#1}}
\newcommand{\DocumentationTok}[1]{\textcolor[rgb]{0.56,0.35,0.01}{\textbf{\textit{#1}}}}
\newcommand{\ErrorTok}[1]{\textcolor[rgb]{0.64,0.00,0.00}{\textbf{#1}}}
\newcommand{\ExtensionTok}[1]{#1}
\newcommand{\FloatTok}[1]{\textcolor[rgb]{0.00,0.00,0.81}{#1}}
\newcommand{\FunctionTok}[1]{\textcolor[rgb]{0.13,0.29,0.53}{\textbf{#1}}}
\newcommand{\ImportTok}[1]{#1}
\newcommand{\InformationTok}[1]{\textcolor[rgb]{0.56,0.35,0.01}{\textbf{\textit{#1}}}}
\newcommand{\KeywordTok}[1]{\textcolor[rgb]{0.13,0.29,0.53}{\textbf{#1}}}
\newcommand{\NormalTok}[1]{#1}
\newcommand{\OperatorTok}[1]{\textcolor[rgb]{0.81,0.36,0.00}{\textbf{#1}}}
\newcommand{\OtherTok}[1]{\textcolor[rgb]{0.56,0.35,0.01}{#1}}
\newcommand{\PreprocessorTok}[1]{\textcolor[rgb]{0.56,0.35,0.01}{\textit{#1}}}
\newcommand{\RegionMarkerTok}[1]{#1}
\newcommand{\SpecialCharTok}[1]{\textcolor[rgb]{0.81,0.36,0.00}{\textbf{#1}}}
\newcommand{\SpecialStringTok}[1]{\textcolor[rgb]{0.31,0.60,0.02}{#1}}
\newcommand{\StringTok}[1]{\textcolor[rgb]{0.31,0.60,0.02}{#1}}
\newcommand{\VariableTok}[1]{\textcolor[rgb]{0.00,0.00,0.00}{#1}}
\newcommand{\VerbatimStringTok}[1]{\textcolor[rgb]{0.31,0.60,0.02}{#1}}
\newcommand{\WarningTok}[1]{\textcolor[rgb]{0.56,0.35,0.01}{\textbf{\textit{#1}}}}
% Add the geometry package and set smaller margins
\usepackage[margin=1in]{geometry}

\usepackage{amsmath,amssymb}
\usepackage{lmodern}
\usepackage{iftex}
\ifPDFTeX
  \usepackage[T1]{fontenc}
  \usepackage[utf8]{inputenc}
  \usepackage{textcomp} % provide euro and other symbols
\else % if luatex or xetex
  \usepackage{unicode-math} % this also loads fontspec
  \defaultfontfeatures{Scale=MatchLowercase}
  \defaultfontfeatures[\rmfamily]{Ligatures=TeX,Scale=1}
\fi
\usepackage{braket}
\usepackage{fancyhdr}
\usepackage{listings}
\usepackage{framed}
\lstnewenvironment{Highlighting}{}{}
\usepackage{xcolor}
% Use upquote if available, for straight quotes in verbatim environments
\IfFileExists{upquote.sty}{\usepackage{upquote}}{}

\IfFileExists{microtype.sty}{
  \usepackage[]{microtype}
  \UseMicrotypeSet[protrusion]{basicmath} % disable protrusion for tt fonts
}{}

\makeatletter
\@ifundefined{KOMAClassName}{% if non-KOMA class
  \IfFileExists{parskip.sty}{
    \usepackage{parskip}
  }{% else
    \setlength{\parindent}{0pt}
    \setlength{\parskip}{6pt plus 2pt minus 1pt}
  }
}{% if KOMA class
  \KOMAoptions{parskip=half}
}
\makeatother

\usepackage{xcolor}
\usepackage{longtable,booktabs,array}
\usepackage{calc} % for calculating minipage widths

% Correct order of tables after \paragraph or \subparagraph
\usepackage{etoolbox}
\makeatletter
\patchcmd\longtable{\par}{\if@noskipsec\mbox{}\fi\par}{}{}
\makeatother

% Allow footnotes in longtable head/foot
\IfFileExists{footnotehyper.sty}{\usepackage{footnotehyper}}{\usepackage{footnote}}
\makesavenoteenv{longtable}

\setlength{\emergencystretch}{3em} % prevent overfull lines
\providecommand{\tightlist}{%
  \setlength{\itemsep}{0pt}\setlength{\parskip}{0pt}}
\setcounter{secnumdepth}{-\maxdimen} % remove section numbering

\usepackage{framed}
\usepackage{listings}

\ifLuaTeX
  \usepackage{selnolig} % disable illegal ligatures
\fi

\IfFileExists{bookmark.sty}{\usepackage{bookmark}}{\usepackage{hyperref}}
\IfFileExists{xurl.sty}{\usepackage{xurl}}{} % add URL line breaks if available
\urlstyle{same}
\hypersetup{
  hidelinks,
  pdfcreator={LaTeX via pandoc}
}

% Custom header and footer
\pagestyle{fancy}
\fancyhead[l]{Azal Amer}
\fancyhead[c]{Modern Physics Midterm  \#1}
\fancyhead[r]{\today}
\fancyfoot[c]{\thepage}
\renewcommand{\headrulewidth}{.2pt}
\setlength{\headheight}{15pt}

\begin{document}
\definecolor{shadecolor}{rgb}{0.9, 0.9, 0.9}

\hypertarget{problem-1}{%
\section{Problem 1}\label{problem-1}}

\begin{enumerate}
\def\labelenumi{\arabic{enumi}.}
\tightlist
\item
  T
\item
  T
\item
  F
\item
  T
\item
  F
\end{enumerate}

\hypertarget{problem-2}{%
\section{Problem 2}\label{problem-2}}

\hypertarget{section}{%
\subsection{1)}\label{section}}

2 possible states, 4 particles, allowed with repeated states, is \[
\Omega = 2^{4} = 32
\]

\hypertarget{section-1}{%
\subsection{2)}\label{section-1}}

\begin{longtable}[]{@{}ll@{}}
\toprule\noalign{}
\# Up & \# Down \\
\midrule\noalign{}
\endhead
\bottomrule\noalign{}
\endlastfoot
4 & 0 \\
3 & 1 \\
2 & 2 \\
1 & 3 \\
0 & 4 \\
\end{longtable}

There are 5 discrete possible energies, as from the principle of
indifference, the particles are indistinguishable from each other. The
question doesn't ask for probability of occurrence, so we can just use
\[
E = \left( N-2N_{\uparrow} \right) \mu B
\]

Back in our table, this gives us

\begin{longtable}[]{@{}lll@{}}
\toprule\noalign{}
\# Up & \# Down & Energy \\
\midrule\noalign{}
\endhead
\bottomrule\noalign{}
\endlastfoot
4 & 0 & \(-4\mu B\) \\
3 & 1 & \(-2\mu B\) \\
2 & 2 & \(0\mu B\) \\
1 & 3 & \(2\mu B\) \\
0 & 4 & \(4\mu B\) \\
\end{longtable}

\hypertarget{section-2}{%
\subsection{3)}\label{section-2}}

\hypertarget{beta}{%
\subsection{\texorpdfstring{\(\beta)\)}{\textbackslash beta)}}\label{beta}}

\begin{quote}
If the system is in thermal equilibrium with a reservoir at temperature
\(T = \mu B / k_B\), what is the probability that the system has total
energy \(-2\mu B\)? (Don't waste much time simplifying exponentials.) (5
pts)
\end{quote}

We know the number of micro states \(\Omega = 32\),and the total
temperature. We are looking for a function which takes

\[
P\left( \Omega=32,T = \frac{\mu B}{k_{B}} \bigg| E_{tot} = -2\mu B \right)
\]

This seems like a canonical ensemble problem, because we don't know
about the conditions of the reservoir? We can use the Boltzmann factor:
\[
p \propto \exp\left( \cancel{ - }\frac{\cancel{ - }2\cancel{ \mu B }}{\frac{\cancel{ k_{B} }\cancel{ \mu B }}{\cancel{ kB }}} \right) = e^{ 2 } 
\] This term is unnormalized, so we need to divide it out by all
possible states \[
p = \frac{e^{2}}{\sum_{i=-2}^{2}e^{ 2i }}
\]

\begin{Shaded}
\begin{Highlighting}[]
\NormalTok{normalizer }\OperatorTok{=} \DecValTok{0}
\NormalTok{constants }\OperatorTok{=}\NormalTok{ [}\DecValTok{1}\NormalTok{,}\DecValTok{4}\NormalTok{,}\DecValTok{6}\NormalTok{,}\DecValTok{4}\NormalTok{,}\DecValTok{1}\NormalTok{]}
\ControlFlowTok{for}\NormalTok{ i }\KeywordTok{in} \BuiltInTok{range}\NormalTok{(}\DecValTok{5}\NormalTok{):}
\NormalTok{    normalizer }\OperatorTok{+=}\NormalTok{constants[i]}\OperatorTok{*}\NormalTok{(np.e}\OperatorTok{**}\NormalTok{(i}\OperatorTok{*}\DecValTok{2}\OperatorTok{{-}}\DecValTok{4}\NormalTok{))}
\BuiltInTok{print}\NormalTok{(normalizer)}

\BuiltInTok{print}\NormalTok{(np.e}\OperatorTok{**}\DecValTok{2}\OperatorTok{/}\NormalTok{normalizer)}
\end{Highlighting}
\end{Shaded}

This works out to about \(\approx 8\%\)

\hypertarget{problem-3}{%
\section{Problem 3}\label{problem-3}}

Isobaric system

\includegraphics{../../../Supplemental Files/images/Pasted image 20240924123857.png}

\hypertarget{a}{%
\subsection{a)}\label{a}}

We can represent this as work done on the system. Like the pressure
applied over the area, over a distance. \[
PV = E
\] The pressure is constant, but the volume changes, so

\[
PV_{2} = E_{2}
\] The difference in energy, would just be \[
\frac{V_{2}}{V_{1}} = \frac{E_{2}}{E_{1}}
\]

There is work being done on the system by the gas, and the pressure is
constant. This is just the pressure being applied over a area (force),
and then the expansion is the distance (Fd = E).

\hypertarget{b}{%
\subsection{b)}\label{b}}

For an ideal gas, we use the Sakur-Tetrode Equation to solve for
entropy.

We have the ratio of energies, how can I take a ratio and convert it to
a difference?
\[ S = k_B N\left[\ln \left(\frac{V}{N}\left(\frac{4\pi m E}{3h^2 N}\right)^{3/2}\right) + \frac{5}{2}\right]\]
The change in entropy becomes \[
\begin{align}
 S_{1} = k_B N\left[\ln \left(\frac{V_{1}}{N}\left(\frac{4\pi m E_{1}}{3h^2 N}\right)^{3/2}\right) + \frac{5}{2}\right] \\
 S_{2} = k_B N\left[\ln \left(\frac{V_{2}}{N}\left(\frac{4\pi m E_{2}}{3h^2 N}\right)^{3/2}\right) + \frac{5}{2}\right] \\
\Delta S = S_{2}-S_{1}
\end{align}
\] We use the additive property of natural logs to expand this \[
\begin{align}
\Delta S = k_{B}N\left[ \ln\left( \frac{V_{1}}{N}+\frac{3}{2} \ln \frac{4\pi mE_{1}}{3h^{2}N}\right) +\cancel{ \frac{5}{2} }-\ln\left( \frac{V_{2}}{N}+\frac{3}{2} \ln \frac{4\pi mE_{2}}{3h^{2}N}\right) -\cancel{ \frac{5}{2} }\right]  \\
\frac{\Delta S}{k_{B}N} = \ln\left( \frac{V_{1}}{N} \right) + \frac{3}{2}\left( \ln E_{1}+\cancel{ \ln\frac{{4}\pi m}{3h^{2}N} } \right)C -\ln\left( \frac{V_{2}}{N} \right) + \frac{3}{2}\left( \ln E_{2}+\cancel{ \ln\frac{{4}\pi m}{3h^{2}N} } \right) \\
=\ln \frac{V_{1}}{N}+\frac{3}{2}\ln E_{1}-\ln \frac{V_{2}}{N}-\frac{3}{2}E_{2} \\
=\left( \ln \frac{V_{1}}{N}-\ln \frac{V_{2}}{N} \right)+\frac{3}{2}(\ln E_{1}-\ln E_{2}) \\
= \ln \left( \frac{V_{1}}{N}* \frac{N}{V_{2}} \right) +\frac{3}{2}\ln\left( \frac{E_{2}}{E_{1}} \right) \\
= -\ln\left( \frac{V_{2}}{V_{1}} \right)+\frac{3}{2}\ln\left( \frac{V_{2}}{V_{1}} \right) \\
\frac{\Delta S}{k_{B}N}=\frac{1}{2}\ln\left( \frac{V_{2}}{V_{1}} \right) \\
\Delta S = \frac{k_{B}nN_{A}}{2}\ln\left( \frac{V_{2}}{V_{1}} \right)
\end{align}
\]

Sanity check wise, it makes sense that this is positive. The dimensions
are matching as well, the Boltzmann constant has units of energy/temp,
same with entropy.
\end{document}