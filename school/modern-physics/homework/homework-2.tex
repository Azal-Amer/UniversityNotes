\PassOptionsToPackage{unicode}{hyperref}
\PassOptionsToPackage{hyphens}{url}

\documentclass[]{article}

% Add the geometry package and set smaller margins
\usepackage[margin=1in]{geometry}

\usepackage{amsmath,amssymb}
\usepackage{lmodern}
\usepackage{iftex}
\ifPDFTeX
  \usepackage[T1]{fontenc}
  \usepackage[utf8]{inputenc}
  \usepackage{textcomp} % provide euro and other symbols
\else % if luatex or xetex
  \usepackage{unicode-math} % this also loads fontspec
  \defaultfontfeatures{Scale=MatchLowercase}
  \defaultfontfeatures[\rmfamily]{Ligatures=TeX,Scale=1}
\fi

\usepackage{fancyhdr}

% Use upquote if available, for straight quotes in verbatim environments
\IfFileExists{upquote.sty}{\usepackage{upquote}}{}

\IfFileExists{microtype.sty}{
  \usepackage[]{microtype}
  \UseMicrotypeSet[protrusion]{basicmath} % disable protrusion for tt fonts
}{}

\makeatletter
\@ifundefined{KOMAClassName}{% if non-KOMA class
  \IfFileExists{parskip.sty}{
    \usepackage{parskip}
  }{% else
    \setlength{\parindent}{0pt}
    \setlength{\parskip}{6pt plus 2pt minus 1pt}
  }
}{% if KOMA class
  \KOMAoptions{parskip=half}
}
\makeatother

\usepackage{xcolor}
\usepackage{longtable,booktabs,array}
\usepackage{calc} % for calculating minipage widths

% Correct order of tables after \paragraph or \subparagraph
\usepackage{etoolbox}
\makeatletter
\patchcmd\longtable{\par}{\if@noskipsec\mbox{}\fi\par}{}{}
\makeatother

% Allow footnotes in longtable head/foot
\IfFileExists{footnotehyper.sty}{\usepackage{footnotehyper}}{\usepackage{footnote}}
\makesavenoteenv{longtable}

\setlength{\emergencystretch}{3em} % prevent overfull lines
\providecommand{\tightlist}{%
  \setlength{\itemsep}{0pt}\setlength{\parskip}{0pt}}
\setcounter{secnumdepth}{-\maxdimen} % remove section numbering

\ifLuaTeX
  \usepackage{selnolig} % disable illegal ligatures
\fi

\IfFileExists{bookmark.sty}{\usepackage{bookmark}}{\usepackage{hyperref}}
\IfFileExists{xurl.sty}{\usepackage{xurl}}{} % add URL line breaks if available
\urlstyle{same}
\hypersetup{
  hidelinks,
  pdfcreator={LaTeX via pandoc}
}

% Custom header and footer
\pagestyle{fancy}
\fancyhead[l]{Azal Amer}
\fancyhead[c]{ HW \#}
\fancyhead[r]{\today}
\fancyfoot[c]{\thepage}
\renewcommand{\headrulewidth}{.2pt}
\setlength{\headheight}{15pt}

\begin{document}
\hypertarget{problem-1}{%
\section{Problem 1}\label{problem-1}}

\[
\sigma = \sqrt{ \frac{p(1-p)}{N} }
\] We need to solve for the number of people to make \(\sigma=.015\) \[
\begin{align}
\frac{.015^{2}}{p(1-p)} = N^{-1} \\
\frac{.015^{2}}{.35*(1-.35)}=N^{-1} \\
\frac{.015^{2}}{.35*.65} = N^{-1}
\end{align}
\]

\begin{Shaded}
\begin{Highlighting}[]
\NormalTok{print(((.015**2)/(.35*.65))**{-}1)}
\end{Highlighting}
\end{Shaded}

\[
N = 1012 \text{ people}
\] \# Problem 2 \[
\begin{align}
\sigma=\sqrt{ \frac{.5^{2}}{N_{A}} } \\
\sigma=\sqrt{ \frac{.25}{6.002\times {10}^{23}} }
\end{align}
\]

\begin{Shaded}
\begin{Highlighting}[]
\NormalTok{N\_A = 6.002e23}
\NormalTok{print((.25/N\_A)**.5)}
\end{Highlighting}
\end{Shaded}

\[
\sigma_{\text{left}} = 6.44\times 10^{-13}
\]

\hypertarget{problem-3}{%
\section{Problem 3}\label{problem-3}}

Given \(O_{2}\) at STP, \(1mol = 22.4L\) The mass of 1 oxygen is
\(5.3*10^{-26}\). What's the rms of the molecules? The rms of the
molecules \[
v_{rms} = \sqrt{ \bar{v}^{2} }
\] We also know from an extension of the Equipartition Principle, that
the variance of particle's velocity in a gas, is equal to the average
velocity in a gas. This means that \(v_{rms} = \sigma_{v}\) Then to
solve for \(\sigma_{v}\), we do \[
\begin{align}
\sigma_{v} = \sqrt{ \frac{{3k_{b}T}}{m} }
\end{align}
\] We don't actually need the number of molecules in this gas, or the
pressure.

\begin{Shaded}
\begin{Highlighting}[]
\NormalTok{m = 5.3e{-}26}
\NormalTok{k\_b = 1.38e{-}23}
\NormalTok{T = 273}
\NormalTok{sigma = np.sqrt(3*k\_b*T/m)}
\NormalTok{print(sigma)}
\end{Highlighting}
\end{Shaded}

\[\sigma_{v} = 461.788 \frac{m}{s}\]

\hypertarget{problem-4}{%
\section{Problem 4}\label{problem-4}}

{[}{[}../../../Supplemental Files/images/Pasted image
20240910131751.png{]}{]} First I made a table with a bounded combination
of the results.

\hypertarget{problem-5}{%
\section{Problem 5}\label{problem-5}}

Find the most likely energy of the Maxwell Boltzmann Distribution \[
f(v) = 4\pi \left(\frac{m}{2\pi kT}\right)^{3/2} v^2 e^{-\frac{mv^2}{2kT}}
\] Since this is a transformed normal distribution, we need to find the
value of \(v(m,T)\) which maximizes the distribution.

To solve this, can't we just take \(\frac{\partial}{\partial v}\)?

\begin{Shaded}
\begin{Highlighting}[]
\ImportTok{import}\NormalTok{ sympy }\ImportTok{as}\NormalTok{ sp}
\CommentTok{\# make a function for maxwell boltzman}
\KeywordTok{def}\NormalTok{ maxwell\_boltzman(v, T, m):}
    \ControlFlowTok{return} \DecValTok{4} \OperatorTok{*}\NormalTok{ sp.pi }\OperatorTok{*}\NormalTok{ (m }\OperatorTok{/}\NormalTok{ (}\DecValTok{2} \OperatorTok{*}\NormalTok{ sp.pi }\OperatorTok{*}\NormalTok{ k }\OperatorTok{*}\NormalTok{ T))}\OperatorTok{**}\NormalTok{(}\DecValTok{3}\OperatorTok{/}\DecValTok{2}\NormalTok{) }\OperatorTok{*}\NormalTok{ v}\OperatorTok{**}\DecValTok{2} \OperatorTok{*}\NormalTok{ sp.exp(}\OperatorTok{{-}}\NormalTok{m }\OperatorTok{*}\NormalTok{ v}\OperatorTok{**}\DecValTok{2} \OperatorTok{/}\NormalTok{ (}\DecValTok{2} \OperatorTok{*}\NormalTok{ k }\OperatorTok{*}\NormalTok{ T))}
\CommentTok{\# define the symbols}
\NormalTok{m }\OperatorTok{=}\NormalTok{ sp.symbols(}\StringTok{\textquotesingle{}m\textquotesingle{}}\NormalTok{)}
\NormalTok{v }\OperatorTok{=}\NormalTok{ sp.symbols(}\StringTok{\textquotesingle{}v\textquotesingle{}}\NormalTok{)}
\NormalTok{T }\OperatorTok{=}\NormalTok{ sp.symbols(}\StringTok{\textquotesingle{}T\textquotesingle{}}\NormalTok{)}
\NormalTok{k }\OperatorTok{=}\NormalTok{ sp.symbols(}\StringTok{\textquotesingle{}k\textquotesingle{}}\NormalTok{)}
\CommentTok{\# print the function}
\NormalTok{maxwell\_boltzman(v, T, m)}
\CommentTok{\# take the partial derivative}
\NormalTok{sp.diff(maxwell\_boltzman(v, T, m), v)}
\CommentTok{\# solve when it equals zero, we can only have one extrema}
\BuiltInTok{print}\NormalTok{(sp.solve(sp.diff(maxwell\_boltzman(v, T, m), v), v))}
\end{Highlighting}
\end{Shaded}

Then is \[
v_{max} = \sqrt{ \frac{2Tk}{m} }
\]
\end{document}