\PassOptionsToPackage{unicode}{hyperref}
\PassOptionsToPackage{hyphens}{url}

\documentclass[]{article}


\usepackage{color}
\usepackage{graphicx}

\usepackage{fancyvrb}
\newcommand{\VerbBar}{|}
\newcommand{\VERB}{\Verb[commandchars=\\\{\}]}
\DefineVerbatimEnvironment{Highlighting}{Verbatim}{commandchars=\\\{\}}
% Add ',fontsize=\small' for more characters per line
\usepackage{framed}
\definecolor{shadecolor}{RGB}{248,248,248}
\newenvironment{Shaded}{\begin{snugshade}}{\end{snugshade}}
\newcommand{\AlertTok}[1]{\textcolor[rgb]{0.94,0.16,0.16}{#1}}
\newcommand{\AnnotationTok}[1]{\textcolor[rgb]{0.56,0.35,0.01}{\textbf{\textit{#1}}}}
\newcommand{\AttributeTok}[1]{\textcolor[rgb]{0.13,0.29,0.53}{#1}}
\newcommand{\BaseNTok}[1]{\textcolor[rgb]{0.00,0.00,0.81}{#1}}
\newcommand{\BuiltInTok}[1]{#1}
\newcommand{\CharTok}[1]{\textcolor[rgb]{0.31,0.60,0.02}{#1}}
\newcommand{\CommentTok}[1]{\textcolor[rgb]{0.56,0.35,0.01}{\textit{#1}}}
\newcommand{\CommentVarTok}[1]{\textcolor[rgb]{0.56,0.35,0.01}{\textbf{\textit{#1}}}}
\newcommand{\ConstantTok}[1]{\textcolor[rgb]{0.56,0.35,0.01}{#1}}
\newcommand{\ControlFlowTok}[1]{\textcolor[rgb]{0.13,0.29,0.53}{\textbf{#1}}}
\newcommand{\DataTypeTok}[1]{\textcolor[rgb]{0.13,0.29,0.53}{#1}}
\newcommand{\DecValTok}[1]{\textcolor[rgb]{0.00,0.00,0.81}{#1}}
\newcommand{\DocumentationTok}[1]{\textcolor[rgb]{0.56,0.35,0.01}{\textbf{\textit{#1}}}}
\newcommand{\ErrorTok}[1]{\textcolor[rgb]{0.64,0.00,0.00}{\textbf{#1}}}
\newcommand{\ExtensionTok}[1]{#1}
\newcommand{\FloatTok}[1]{\textcolor[rgb]{0.00,0.00,0.81}{#1}}
\newcommand{\FunctionTok}[1]{\textcolor[rgb]{0.13,0.29,0.53}{\textbf{#1}}}
\newcommand{\ImportTok}[1]{#1}
\newcommand{\InformationTok}[1]{\textcolor[rgb]{0.56,0.35,0.01}{\textbf{\textit{#1}}}}
\newcommand{\KeywordTok}[1]{\textcolor[rgb]{0.13,0.29,0.53}{\textbf{#1}}}
\newcommand{\NormalTok}[1]{#1}
\newcommand{\OperatorTok}[1]{\textcolor[rgb]{0.81,0.36,0.00}{\textbf{#1}}}
\newcommand{\OtherTok}[1]{\textcolor[rgb]{0.56,0.35,0.01}{#1}}
\newcommand{\PreprocessorTok}[1]{\textcolor[rgb]{0.56,0.35,0.01}{\textit{#1}}}
\newcommand{\RegionMarkerTok}[1]{#1}
\newcommand{\SpecialCharTok}[1]{\textcolor[rgb]{0.81,0.36,0.00}{\textbf{#1}}}
\newcommand{\SpecialStringTok}[1]{\textcolor[rgb]{0.31,0.60,0.02}{#1}}
\newcommand{\StringTok}[1]{\textcolor[rgb]{0.31,0.60,0.02}{#1}}
\newcommand{\VariableTok}[1]{\textcolor[rgb]{0.00,0.00,0.00}{#1}}
\newcommand{\VerbatimStringTok}[1]{\textcolor[rgb]{0.31,0.60,0.02}{#1}}
\newcommand{\WarningTok}[1]{\textcolor[rgb]{0.56,0.35,0.01}{\textbf{\textit{#1}}}}
% Add the geometry package and set smaller margins
\usepackage[margin=1in]{geometry}

\usepackage{amsmath,amssymb}
\usepackage{lmodern}
\usepackage{iftex}
\ifPDFTeX
  \usepackage[T1]{fontenc}
  \usepackage[utf8]{inputenc}
  \usepackage{textcomp} % provide euro and other symbols
\else % if luatex or xetex
  \usepackage{unicode-math} % this also loads fontspec
  \defaultfontfeatures{Scale=MatchLowercase}
  \defaultfontfeatures[\rmfamily]{Ligatures=TeX,Scale=1}
\fi
\usepackage{braket}
\usepackage{fancyhdr}
\usepackage{listings}
\usepackage{framed}
\lstnewenvironment{Highlighting}{}{}
\usepackage{xcolor}
% Use upquote if available, for straight quotes in verbatim environments
\IfFileExists{upquote.sty}{\usepackage{upquote}}{}

\IfFileExists{microtype.sty}{
  \usepackage[]{microtype}
  \UseMicrotypeSet[protrusion]{basicmath} % disable protrusion for tt fonts
}{}

\makeatletter
\@ifundefined{KOMAClassName}{% if non-KOMA class
  \IfFileExists{parskip.sty}{
    \usepackage{parskip}
  }{% else
    \setlength{\parindent}{0pt}
    \setlength{\parskip}{6pt plus 2pt minus 1pt}
  }
}{% if KOMA class
  \KOMAoptions{parskip=half}
}
\makeatother

\usepackage{xcolor}
\usepackage{longtable,booktabs,array}
\usepackage{calc} % for calculating minipage widths

% Correct order of tables after \paragraph or \subparagraph
\usepackage{etoolbox}
\makeatletter
\patchcmd\longtable{\par}{\if@noskipsec\mbox{}\fi\par}{}{}
\makeatother

% Allow footnotes in longtable head/foot
\IfFileExists{footnotehyper.sty}{\usepackage{footnotehyper}}{\usepackage{footnote}}
\makesavenoteenv{longtable}

\setlength{\emergencystretch}{3em} % prevent overfull lines
\providecommand{\tightlist}{%
  \setlength{\itemsep}{0pt}\setlength{\parskip}{0pt}}
\setcounter{secnumdepth}{-\maxdimen} % remove section numbering

\usepackage{framed}
\usepackage{listings}

\ifLuaTeX
  \usepackage{selnolig} % disable illegal ligatures
\fi

\IfFileExists{bookmark.sty}{\usepackage{bookmark}}{\usepackage{hyperref}}
\IfFileExists{xurl.sty}{\usepackage{xurl}}{} % add URL line breaks if available
\urlstyle{same}
\hypersetup{
  hidelinks,
  pdfcreator={LaTeX via pandoc}
}

% Custom header and footer
\pagestyle{fancy}
\fancyhead[l]{Azal Amer}
\fancyhead[c]{Modern Physics HW \#3}
\fancyhead[r]{\today}
\fancyfoot[c]{\thepage}
\renewcommand{\headrulewidth}{.2pt}
\setlength{\headheight}{15pt}

\begin{document}
\definecolor{shadecolor}{rgb}{0.9, 0.9, 0.9}

\hypertarget{problem-1}{%
\section{Problem 1}\label{problem-1}}

\hypertarget{a}{%
\subsection{a)}\label{a}}

The system will not spontaneously reverse, because systems can only
spontaneously move to areas of equal or higher numbers of micro states.

\hypertarget{b}{%
\subsection{b)}\label{b}}

\[
\ln\Omega= \ln\Omega_{He}+\ln\Omega_{H}
\]

\[
\Delta S = -nR\left( \ln \frac{V_{1}}{V_{tot}}+\ln \frac{V_{2}}{V_{tot}} \right) 
\] We know that both halves are half the total, which means that this
turns into \[
\Delta S=2nR\left( \ln{2} \right) 
\] \[
n = 2N_{A}
\] Plugging this all in gives a final entropy of'' \[
\Delta S = 1.725e24
\]

\hypertarget{problem-2}{%
\section{Problem 2}\label{problem-2}}

\hypertarget{a-1}{%
\subsection{a)}\label{a-1}}

The number of permutations of a system given a fixed value of known
states can be given by the binomial coefficient. \[
\Omega(N_{\uparrow})=\begin{pmatrix}
N \\
N_{\uparrow}
\end{pmatrix} = \frac{{N!}}{N_{\uparrow}!\left( N-N_{\uparrow} \right) !}
\] \#\# b)

\$\$ \begin{align}
\Omega \left( N_{\uparrow} \right)  = \frac{{N!}}{N_{\uparrow}!\left( N-N_{\uparrow} \right) !} \\
\ln(\Omega \left( N_{\uparrow} \right) ) = \ln \left( \frac{{N!}}{N_{\uparrow}!\left( N-N_{\uparrow} \right) !} \right)  \\
=\frac{\ln(N!)}{\ln(N_\uparrow(N-N_{\uparrow})!)} \\
= \frac{\ln(N!)}{\ln(N_{\uparrow}!)+\ln((N-N_{\uparrow})!)} \\

= \ln(N!)-\ln(N_{\uparrow}!)-\ln((N-N_{\uparrow})!) \\
\approx N\ln N-N+N_{\uparrow}\ln N_{\uparrow}-N_{\uparrow}+(N-N_{\uparrow})\ln(N-N_{\uparrow})-(N-N_{\uparrow}) \\
\ln\Omega(N_{\uparrow})\approx N\ln N-N_{\uparrow}\ln N_{\uparrow}-(N-N_{\uparrow})\ln(N-N_{\uparrow})
\end{align} \[
## c)
The energy of the spin-system is given by
\] E = (2N\_\{\uparrow\}-N)\mu B \[
Solving for $N_{\uparrow}$ 
\] N\_\{\uparrow\} = \frac{\frac{E}{\mu B}-N}{2} \$\$ Then plugging in
this value of \(N_{\uparrow}\) to the previous expression, gets us

\[
\ln\Omega(E) = N\ln N-N-\frac{1}{2}\left( \frac{E}{\mu B}-N \right)\ln\left( \frac{E}{\mu B}-N \right)-\left( \frac{E+N}{2\mu B} \right) \ln\left( \frac{E+N}{2\mu B} \right) 
\]

\hypertarget{d}{%
\subsection{d)}\label{d}}

\begin{Shaded}
\begin{Highlighting}[]
\NormalTok{import sympy as sp}
\NormalTok{E, N, mu, B = sp.symbols(\textquotesingle{}E N mu B\textquotesingle{})}
\NormalTok{ln\_Omega = N * sp.log(N) {-} N {-} sp.Rational(1,2) * ((E/(mu*B) {-} N) * sp.log(E/(mu*B) {-} N)) {-} ((E + N)/(2*mu*B)) * sp.log((E + N)/(2*mu*B))}
\NormalTok{print(sp.latex(sp.simplify(sp.diff(ln\_Omega,E)**{-}1)))}
\end{Highlighting}
\end{Shaded}

\[
T = - \frac{2 B \mu}{\ln{\left(\frac{E + N}{2 B \mu} \right)} + \ln{\left(- \frac{B N \mu - E}{B \mu} \right)} + 2}
\] \# Problem 3

\begin{longtable}[]{@{}
  >{\raggedright\arraybackslash}p{(\columnwidth - 0\tabcolsep) * \real{1.0000}}@{}}
\toprule\noalign{}
\begin{minipage}[b]{\linewidth}\raggedright
{[}{[}../../../Supplemental Files/images/Homework 3 2024-09-18
16.39.13.excalidraw{]}{]}
\end{minipage} \\
\midrule\noalign{}
\endhead
\bottomrule\noalign{}
\endlastfoot
Diagram of the System \\
\end{longtable}

The temperature is the same For either system, the entropy is given by

\[
S = nR \left[ \ln \left( \frac{V E^{3/2}}{n^{5/2}} \right) + \frac{5}{2} \right]
\] The total entropy of the systems is \[
S_{total} = n_{1}R \left[ \ln \left( \frac{V_{1} E_{1}^{3/2}}{n_{1}^{5/2}} \right) + \frac{5}{2} \right]+n_{2}R \left[ \ln \left( \frac{V_{1} E_{2}^{3/2}}{n_{2}^{5/2}} \right) + \frac{5}{2} \right]
\]

Entropy is maximized when \[
\frac{dS_{total}}{dE}=0
\] Total Energy is defined as \[
E= E_{1}+E_{2}
\] We rearrange, and differentiate with respect to \(E_{1}\) \[
E_{1} = E-E_{2}
\] \[
\frac{dS_{tot}}{dE_{1}} = 0
\]

\begin{Shaded}
\begin{Highlighting}[]
\NormalTok{\# Define the symbols}
\NormalTok{n1, n2, R, V1, V2, E1, E2 = sp.symbols(\textquotesingle{}n1 n2 R V1 V2 E1 E2\textquotesingle{})}

\NormalTok{\# Sackur{-}Tetrode equation for the total entropy}
\NormalTok{S\_total = n1 * R * (sp.log((V1 * E1**(3/2)) / n1**(5/2)) + 5/2) + n2 * R * (sp.log((V2 * E2**(3/2)) / n2**(5/2)) + 5/2)}
\NormalTok{ans =S\_total.subs(\{E2:1{-}E1\}).diff(E1)}

\NormalTok{print(sp.latex(ans))}
\end{Highlighting}
\end{Shaded}

\[
\begin{align}
\frac{dS_{tot}}{dE_{1}}  =3\frac{n_{1}R}{2E_{1}}- \frac{3}{2}\frac{n_{2}R}{E_{1}} \\
=0 \\
\frac{n_{1}}{E_{1}}=\frac{n_{2}}{E_{2}} \\
\frac{E_{1}}{E_{2}} = \frac{n_{1}}{n_{2}}
\end{align}
\]

\hypertarget{problem-4}{%
\section{Problem 4}\label{problem-4}}

\emph{Credit to Austin Merkel for helping me with this problem!} \#\# a)
Energy is a function of temperature, which is not conserved in this
process. This means that as a gas expands, the energy will drop We can
obtain this from \[
\frac{\Delta E}{\Delta S} = T
\] This above shows us this energy drop, but we want to quantitatively
describe the shift. We can use the Sackur-Tetrode equation. \[
S\propto \ln(V_{i}E_{i}^{3/2})
\] The entropy of both forms of the system is identical, so \[
\begin{align}
\ln(V_{1}E_{1}^{3/2}) = \ln(V_{2}E_{2}^{3/2}) \\
V_{1}E_{1}^{3/2} = V_{2}E_{2}^{3/2} \\
E_{1} = \left( \frac{V_{2}}{V_{1}}^{2/3} \right)E_{2}
\end{align}
\]

\$\$ \begin{align}

\end{align} \$\$

\hypertarget{b-1}{%
\subsection{b)}\label{b-1}}

We know that \[
P = \frac{2}{3}\frac{N}{V}E
\] From part A, we plug in it's values of energies \[
\begin{align}
P_{1}= \frac{2}{3} \frac{N}{V_{1}}*\left( \frac{V_{2}}{V_{1}} \right)^{2/3}E_{2} \\
P_{2}= \frac{2}{3} \frac{N}{V_{2}}*\left( \frac{V_{2}}{V_{1}} \right)^{2/3}E_{1}
\end{align}
\] Which then gives us \[
\begin{align}
\frac{P_{1}}{P_{2}} = V_{2}^{5/3}V_{1}^{-5/3} \\
P_{1}V_{1}^{5/3} = P_{2}V_{2}^{5/3}
\end{align}
\]
\end{document}